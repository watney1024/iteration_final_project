%!TeX program = xelatex
\documentclass[12pt,hyperref,a4paper,UTF8]{ctexart}
\usepackage{SJTUReport}

%%-------------------------------正文开始---------------------------%%
\begin{document}

%%-----------------------封面--------------------%%
\cover

%%------------------摘要-------------%%
%\begin{abstract}
%
%在此填写摘要内容
%
%\end{abstract}

\thispagestyle{empty} % 首页不显示页码

%%--------------------------目录页------------------------%%
\newpage
\tableofcontents

%%------------------------正文页从这里开始-------------------%
\newpage

\section{神经网络算子并行化}

神经网络中的算子计算通常具有高度的数据并行性,通过OpenMP等并行技术可以显著提升计算性能。本节介绍两个典型的神经网络算子:卷积(Convolution)和平均池化(Average Pooling)的并行实现。

\subsection{计算密集型 vs 访存密集型}

在并行计算中,算子的性能特征可以分为两类:

\subsubsection{计算密集型(Compute-Bound)}

\textbf{定义:} 算法的性能瓶颈在于计算单元(ALU、FPU)的吞吐量,而非内存带宽。

\textbf{特征:}
\begin{itemize}
    \item \textbf{高计算访存比(Compute-to-Memory Ratio)}:每次内存访问对应大量计算操作
    \item \textbf{算术强度大}:浮点运算次数 / 内存访问字节数 >> 1
    \item \textbf{瓶颈}:处理器的浮点运算能力
    \item \textbf{并行效果}:通常能获得较好的加速比
\end{itemize}

\textbf{典型例子:}
\begin{itemize}
    \item \textbf{卷积操作}:每个输出元素需要 $C_{in} \times K_h \times K_w$ 次乘加运算
    \begin{itemize}
        \item 对于5×5卷积,32输入通道:每个输出需要 $32 \times 5 \times 5 = 800$ 次乘加
        \item 算术强度:$800 \text{ FLOP} / (800 \times 4 \text{ bytes}) \approx 0.25$ FLOP/byte
    \end{itemize}
    \item \textbf{矩阵乘法}:$O(n^3)$ 计算量,$O(n^2)$ 内存访问
\end{itemize}

\textbf{优化策略:}
\begin{itemize}
    \item 增加并行度(多线程、SIMD)
    \item 提高指令级并行(ILP)
    \item 循环展开、流水线优化
\end{itemize}

\subsubsection{访存密集型(Memory-Bound)}

\textbf{定义:} 算法的性能瓶颈在于内存系统的带宽,而非计算能力。

\textbf{特征:}
\begin{itemize}
    \item \textbf{低计算访存比}:每次内存访问对应很少的计算操作
    \item \textbf{算术强度小}:浮点运算次数 / 内存访问字节数 << 1
    \item \textbf{瓶颈}:内存带宽、缓存命中率
    \item \textbf{并行效果}:容易受内存带宽限制,加速比受限
\end{itemize}

\textbf{典型例子:}
\begin{itemize}
    \item \textbf{池化操作}:每个输出元素只需要 $K_h \times K_w$ 次加法和1次除法
    \begin{itemize}
        \item 对于2×2池化:4次加法 + 1次除法 = 5次操作
        \item 算术强度:$5 \text{ FLOP} / (4 \times 4 \text{ bytes}) \approx 0.31$ FLOP/byte
    \end{itemize}
    \item \textbf{Batch Normalization}:主要是内存读写和简单运算
    \item \textbf{激活函数}(ReLU等):逐元素操作,计算量极小
\end{itemize}

\textbf{优化策略:}
\begin{itemize}
    \item 提高缓存命中率(数据重用、分块)
    \item 减少内存访问次数
    \item 预取(Prefetch)优化
    \item 内存访问模式优化(连续访问、对齐)
\end{itemize}

\subsubsection{Roofline模型}

性能上限由两个因素决定:

$$
\text{Performance} = \min(\text{Peak FLOPS}, \text{Arithmetic Intensity} \times \text{Memory Bandwidth})
$$

\begin{itemize}
    \item \textbf{计算密集型}:接近Peak FLOPS(屋顶的水平部分)
    \item \textbf{访存密集型}:受限于Memory Bandwidth(屋顶的斜线部分)
\end{itemize}

\textbf{对并行化的影响:}

\begin{table}[!htbp]
    \centering
    \begin{tabular}{|l|l|l|l|}
    \hline
        类型 & 并行加速比 & 主要挑战 & 优化重点 \\
        \hline
        计算密集型 & 接近线性 & 负载均衡 & 增加计算并行度 \\
        访存密集型 & 受限 & 内存带宽竞争 & 减少内存访问、提高缓存效率 \\
        \hline
    \end{tabular}
    \caption{计算密集型与访存密集型对比}
\end{table}

\subsection{卷积算子(Conv2d) - 计算密集型}

\subsubsection{算法原理}

二维卷积是深度学习中最基础且最重要的操作之一。给定输入特征图 $X \in \mathbb{R}^{C_{in} \times H_{in} \times W_{in}}$ 和卷积核 $W \in \mathbb{R}^{C_{out} \times C_{in} \times K_h \times K_w}$,卷积操作计算输出特征图 $Y \in \mathbb{R}^{C_{out} \times H_{out} \times W_{out}}$:

$$
Y[o, h, w] = b[o] + \sum_{c=0}^{C_{in}-1} \sum_{i=0}^{K_h-1} \sum_{j=0}^{K_w-1} X[c, h \cdot s_h + i, w \cdot s_w + j] \cdot W[o, c, i, j]
$$

其中:
\begin{itemize}
    \item $C_{in}$, $C_{out}$: 输入和输出通道数
    \item $K_h$, $K_w$: 卷积核高度和宽度
    \item $s_h$, $s_w$: 步长(stride)
    \item $b[o]$: 偏置项
\end{itemize}

输出特征图的尺寸计算公式:

$$
H_{out} = \lfloor \frac{H_{in} + 2 \cdot padding - K_h}{s_h} \rfloor + 1
$$

$$
W_{out} = \lfloor \frac{W_{in} + 2 \cdot padding - K_w}{s_w} \rfloor + 1
$$

\subsubsection{性能测试结果}

测试配置:
\begin{itemize}
    \item 输入尺寸:(1, 3, 150, 150)
    \item 输出尺寸:(1, 32, 150, 150)
    \item 卷积核:5×5, stride=1, padding=2
\end{itemize}

\subsection{平均池化算子(AvgPool2d) - 访存密集型}

\subsubsection{算法原理}

平均池化(Average Pooling)是一种降采样操作,通过计算局部区域的平均值来减小特征图尺寸。给定输入特征图 $X \in \mathbb{R}^{C \times H_{in} \times W_{in}}$ 和池化窗口大小 $(K_h, K_w)$,平均池化计算输出 $Y \in \mathbb{R}^{C \times H_{out} \times W_{out}}$:

$$
Y[c, h, w] = \frac{1}{K_h \times K_w} \sum_{i=0}^{K_h-1} \sum_{j=0}^{K_w-1} X[c, h \cdot s_h + i, w \cdot s_w + j]
$$

其中:
\begin{itemize}
    \item $c$: 通道索引(池化操作独立作用于每个通道)
    \item $s_h$, $s_w$: 步长参数
    \item $K_h$, $K_w$: 池化窗口大小
\end{itemize}

输出尺寸计算:

$$
H_{out} = \lfloor \frac{H_{in} - K_h}{s_h} \rfloor + 1
$$

$$
W_{out} = \lfloor \frac{W_{in} - K_w}{s_w} \rfloor + 1
$$

\textbf{特点:}
\begin{itemize}
    \item 保持通道数不变
    \item 减小空间维度
    \item 增强特征的平移不变性
    \item 降低计算量和参数量
\end{itemize}

\subsubsection{性能测试结果}

测试配置:
\begin{itemize}
    \item 输入尺寸:(1, 320, 300, 300)
    \item 输出尺寸:(1, 320, 150, 150)
    \item 池化窗口:2×2, stride=2
\end{itemize}

\section{红黑排序Gauss-Seidel迭代法}

\subsection{算法原理}

Gauss-Seidel方法是求解线性方程组 $Ax = b$ 的经典迭代方法,常用于求解泊松方程等偏微分方程的离散化系统。对于二维泊松方程:

$$
-\nabla^2 u = f, \quad (x, y) \in \Omega
$$

使用有限差分离散化后得到:

$$
\frac{u_{i-1,j} + u_{i+1,j} + u_{i,j-1} + u_{i,j+1} - 4u_{i,j}}{h^2} = f_{i,j}
$$

标准Gauss-Seidel迭代格式:

$$
u_{i,j}^{(k+1)} = \frac{1}{4}(u_{i-1,j}^{(k+1)} + u_{i+1,j}^{(k)} + u_{i,j-1}^{(k+1)} + u_{i,j+1}^{(k)} - h^2 f_{i,j})
$$

\textbf{数据依赖问题:} 标准Gauss-Seidel方法在更新 $u_{i,j}$ 时依赖于已更新的 $u_{i-1,j}$ 和 $u_{i,j-1}$,这种数据依赖使得算法难以直接并行化。

\subsection{红黑排序(Red-Black Ordering)}

红黑排序将网格点分为两类:
\begin{itemize}
    \item \textbf{红点}:$i + j$ 为偶数
    \item \textbf{黑点}:$i + j$ 为奇数
\end{itemize}

关键性质:
\begin{itemize}
    \item 红点只依赖于黑点,黑点只依赖于红点
    \item 所有红点可以同时更新(并行)
    \item 所有黑点可以同时更新(并行)
\end{itemize}

\subsection{性能测试结果}

测试配置:
\begin{itemize}
    \item 网格规模:512×512×512 (约1.34亿个网格点)
    \item 最大迭代次数:100
    \item 收敛容差:1e-6
    \item 测试环境:8核处理器
\end{itemize}

\textbf{实测结果:}

\begin{table}[!htbp]
    \centering
    \small
    \begin{tabular}{|l|c|c|c|c|c|c|c|}
    \hline
        方法 & 线程数 & 执行时间(s) & 每次迭代(ms) & 加速比 & 并行效率 & 迭代次数 & 最终残差 \\
        \hline
        串行红黑 & - & 36.226 & 362.26 & 1.00× & - & 100 & 1.71e+05 \\
        并行红黑 & 1 & 108.871 & 1088.71 & 1.00× & 100\% & 100 & 1.71e+05 \\
        并行红黑 & 2 & 63.867 & 638.67 & 1.70× & 85.2\% & 100 & 1.71e+05 \\
        并行红黑 & 4 & 41.297 & 412.97 & 2.64× & 65.9\% & 100 & 1.71e+05 \\
        并行红黑 & 8 & 29.504 & 295.04 & 3.69× & 46.1\% & 100 & 1.71e+05 \\
        \hline
    \end{tabular}
    \caption{Gauss-Seidel方法性能测试结果}
\end{table}

\textbf{性能分析:}

\begin{enumerate}
    \item \textbf{加速比分析}:
    \begin{itemize}
        \item 2线程:1.70×,接近理论值2×,并行效率85.2\%
        \item 4线程:2.64×,并行效率65.9\%
        \item 8线程:3.69×,并行效率46.1\%,受内存带宽限制
    \end{itemize}
    
    \item \textbf{串行红黑 vs 并行红黑(1线程)}:
    \begin{itemize}
        \item 串行版本(36.2s)明显快于并行版本单线程(108.9s)
        \item 原因:并行版本的线程管理和同步开销
        \item 说明:只有在多线程时并行版本才有优势
    \end{itemize}
    
    \item \textbf{扩展性瓶颈}:
    \begin{itemize}
        \item 内存带宽:512³规模需要约8GB内存,8线程并发访问导致带宽饱和
        \item 同步开销:每次迭代需要2次同步(红点+黑点)
        \item 缓存一致性:多线程共享数据导致缓存失效
    \end{itemize}
    
    \item \textbf{实际应用价值}:
    \begin{itemize}
        \item 对于超大规模问题(512³),8线程仍可获得3.69×加速
        \item 相比串行红黑(36.2s),8线程并行(29.5s)仍有显著提升
    \end{itemize}
\end{enumerate}

\section{三对角方程组并行求解}

\subsection{问题描述}

三对角线性方程组形式:

$$
\begin{bmatrix}
b_0 & c_0 & & & \\
a_1 & b_1 & c_1 & & \\
& a_2 & b_2 & c_2 & \\
& & \ddots & \ddots & \ddots \\
& & & a_{n-1} & b_{n-1}
\end{bmatrix}
\begin{bmatrix}
x_0 \\ x_1 \\ x_2 \\ \vdots \\ x_{n-1}
\end{bmatrix}
=
\begin{bmatrix}
d_0 \\ d_1 \\ d_2 \\ \vdots \\ d_{n-1}
\end{bmatrix}
$$

三对角方程组广泛出现在:
\begin{itemize}
    \item 一维热传导方程
    \item 样条插值
    \item 有限差分法
    \item 信号处理
\end{itemize}

\subsection{Thomas算法(追赶法)}

\textbf{串行Thomas算法}是求解三对角方程组的经典方法,时间复杂度 $O(n)$。

\textbf{前向消元(Forward Elimination):}

$$
\gamma_0 = \frac{c_0}{b_0}, \quad \rho_0 = \frac{d_0}{b_0}
$$

对于 $i = 1, 2, \ldots, n-1$:

$$
\gamma_i = \frac{c_i}{b_i - a_i \gamma_{i-1}}, \quad \rho_i = \frac{d_i - a_i \rho_{i-1}}{b_i - a_i \gamma_{i-1}}
$$

\textbf{回代(Back Substitution):}

$$
x_{n-1} = \rho_{n-1}
$$

对于 $i = n-2, n-3, \ldots, 0$:

$$
x_i = \rho_i - \gamma_i x_{i+1}
$$

\textbf{数据依赖:} Thomas算法存在严重的数据依赖:
\begin{itemize}
    \item 前向消元:每一步依赖上一步的结果
    \item 回代:从后向前依赖
\end{itemize}

\subsection{性能测试结果}

测试配置:多种规模的三对角系统,测试环境:8核处理器

\subsubsection{测试数据集:test\_8k.txt (n=8,192)}

\begin{table}[!htbp]
    \centering
    \small
    \begin{tabular}{|l|c|c|c|c|c|}
    \hline
        算法版本 & 线程数 & 执行时间(ms) & 加速比 & 并行效率 & 最大残差 \\
        \hline
        串行Thomas & - & 0.148 & 1.00× & - & 3.55e-15 \\
        Brugnano内部串行 & - & 0.296 & 0.50× & - & 3.55e-15 \\
        Brugnano & 1 & 0.261 & 1.13× & - & 2.94e-01 \\
        Brugnano & 2 & 1.070 & 0.28× & 14.0\% & 1.02e+00 \\
        Brugnano & 4 & 0.766 & 0.39× & 9.7\% & 1.53e+00 \\
        Brugnano & 8 & 1.007 & 0.29× & 3.7\% & 3.39e+00 \\
        \hline
    \end{tabular}
    \caption{三对角求解性能测试 (8k)}
\end{table}

\subsubsection{测试数据集:test\_1024k.txt (n=1,048,576) - 最大规模}

\begin{table}[!htbp]
    \centering
    \small
    \begin{tabular}{|l|c|c|c|c|c|}
    \hline
        算法版本 & 线程数 & 执行时间(ms) & 加速比 & 并行效率 & 最大残差 \\
        \hline
        串行Thomas & - & 10.931 & 1.00× & - & 5.33e-15 \\
        Brugnano内部串行 & - & 11.186 & 0.98× & - & 5.33e-15 \\
        Brugnano & 1 & 19.341 & 0.58× & - & 4.42e-02 \\
        Brugnano & 2 & 12.273 & 0.91× & 45.7\% & 1.33e+00 \\
        Brugnano & 4 & 8.589 & 1.30× & 32.6\% & 2.68e+00 \\
        Brugnano & 8 & 8.037 & 1.39× & 17.4\% & 2.83e+00 \\
        \hline
    \end{tabular}
    \caption{三对角求解性能测试 (1024k)}
\end{table}

\textbf{性能分析:}

\begin{enumerate}
    \item \textbf{小规模数据(8k-16k)的问题}:
    \begin{itemize}
        \item 串行Thomas算法最快(0.148-0.253ms)
        \item 并行版本反而更慢,线程开销 > 计算收益
    \end{itemize}
    
    \item \textbf{大规模数据(1024k)}:
    \begin{itemize}
        \item \textbf{并行优势终于显现}:
        \item 4线程:8.589ms,1.30×加速比
        \item 8线程:8.037ms,1.39×加速比
        \item 原因:计算量足够大,并行收益超过开销
    \end{itemize}
    
    \item \textbf{数值稳定性问题}:
    \begin{itemize}
        \item 注意到Brugnano并行版本的最大残差显著增大
        \item 串行Thomas:$10^{-15}$ 级别(机器精度)
        \item 并行Brugnano:$10^0$ 级别(相对误差约1)
        \item 对于工程应用,这些误差($10^0$量级)可能需要进一步优化
    \end{itemize}
    
    \item \textbf{关键结论}:
    \begin{itemize}
        \item \textbf{三对角方程组的并行化具有挑战性}
        \item 只有在规模极大($n > 10^6$)时才值得并行
        \item Amdahl定律体现明显:规约系统求解成为串行瓶颈
    \end{itemize}
\end{enumerate}

\section{总结与分析}

\subsection{实测性能分析}

\subsubsection{Gauss-Seidel方法的成功案例}

\textbf{优势:}
\begin{itemize}
    \item 大规模问题(512³ ≈ 1.34亿网格点)
    \item 计算密集型,每个网格点需要多次迭代
    \item 红黑排序有效打破数据依赖
\end{itemize}

\textbf{实测表现:}
\begin{itemize}
    \item 8线程达到3.69×加速比
    \item 相比串行红黑(36.2s),8线程并行(29.5s)提速18.5\%
    \item 并行效率:2线程85.2\%,4线程65.9\%,8线程46.1\%
\end{itemize}

\textbf{瓶颈分析:}
\begin{itemize}
    \item 内存带宽:8GB数据的并发访问
    \item 同步点:每次迭代2次(红点+黑点)
    \item 缓存一致性开销
\end{itemize}

\subsubsection{三对角求解的挑战}

\textbf{困难:}
\begin{itemize}
    \item 强数据依赖:每一步依赖前一步
    \item 规约系统求解:串行瓶颈
    \item 额外开销:线程管理、数据重组
\end{itemize}

\textbf{实测表现:}
\begin{itemize}
    \item 小规模(8k-16k):并行版本\textbf{更慢},开销 > 收益
    \item 中等规模(128k):勉强持平,4线程1.14×
    \item 大规模(1024k):\textbf{终于盈利},8线程1.39×
\end{itemize}

\textbf{关键发现:}
\begin{itemize}
    \item 并行只在n>100万时才有优势
    \item Amdahl定律的经典体现
    \item 数值误差增大(残差从$10^{-15}$增至$10^0$)
\end{itemize}

\subsection{结论}

通过本项目的实现和测试,我们得出以下结论:

\begin{enumerate}
    \item \textbf{并行化不是万能药}
    \begin{itemize}
        \item 三对角求解:大部分情况串行更快
        \item 只有超大规模(n>100万)才值得并行
    \end{itemize}
    
    \item \textbf{打破数据依赖是关键}
    \begin{itemize}
        \item Gauss-Seidel红黑排序:成功案例
        \item 三对角Brugnano:部分成功
    \end{itemize}
    
    \item \textbf{实际加速比远低于理论值}
    
    主要原因:
    \begin{itemize}
        \item \textbf{Amdahl定律}:串行部分限制(规约系统)
        \item \textbf{内存带宽}:8线程时带宽饱和(Gauss-Seidel 46\%效率)
        \item \textbf{同步开销}:每次迭代需要等待
        \item \textbf{缓存效应}:False sharing, 缓存失效
    \end{itemize}
    
    \item \textbf{规模决定策略}
    \begin{itemize}
        \item $n < 10^4$: 串行优先
        \item $10^4 < n < 10^6$: 视情况而定,测试决策
        \item $n > 10^6$: 并行有显著优势
    \end{itemize}
    
    \item \textbf{数值稳定性需注意}
    \begin{itemize}
        \item Brugnano残差从$10^{-15}$增至$10^0$
        \item 浮点运算顺序改变导致误差累积
        \item 工程应用需要额外验证
    \end{itemize}
\end{enumerate}

\subsection{实验总结}

\begin{table}[!htbp]
    \centering
    \begin{tabular}{|l|l|}
    \hline
        \textbf{项目} & \textbf{结果} \\
        \hline
        测试算法 & 3类(神经网络、迭代法、递推算法) \\
        测试规模 & 从8k到512³(1.34亿) \\
        最佳加速比 & 3.69× (Gauss-Seidel 8线程) \\
        最差情况 & 0.28× (三对角8k数据2线程) \\
        成功案例 & Gauss-Seidel大规模问题 \\
        失败案例 & 三对角小规模问题 \\
        \hline
    \end{tabular}
    \caption{实验总结}
\end{table}

\textbf{核心启示:} 并行计算需要权衡计算收益与并行开销,只有在问题规模足够大、数据依赖可打破的情况下,才能获得有效加速。盲目并行化可能适得其反。

%%----------- 参考文献 -------------------%%
%在reference.bib文件中填写参考文献,此处自动生成

\reference


\end{document}
