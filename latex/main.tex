% !TeX program = xelatex
\documentclass[12pt,hyperref,a4paper,UTF8]{ctexart}
\usepackage{SJTUReport}
\usepackage{booktabs}  % 在导言区添加
\usepackage[commandnameprefix=always]{changes} % final 可以替换为 draft 查看标注效果
\setaddedmarkup{\textcolor{red}{#1}} % 设置新增内容为红色
\setdeletedmarkup{\textcolor{red}{\sout{#1}}} % 设置删除内容为红色并加删除线
\let\comment\chcomment
\usepackage{listings}
\usepackage{xcolor}

% 自定义样式(可选,重点配置)
\lstset{
    language=C++,           % 指定语言
    basicstyle=\ttfamily\small,  % 基础字体:等宽+小号
    keywordstyle=\color{blue}\bfseries,  % 关键字:蓝色+粗体
    commentstyle=\color{gray}\itshape,   % 注释:灰色+斜体
    stringstyle=\color{red},            % 字符串:红色
    numbers=left,            % 行号显示在左侧
    numberstyle=\tiny\color{gray},  % 行号样式:小号+灰色
    frame=single,            % 单框包围代码
    frameround=ffff,         % 边框圆角(可选)
    breaklines=true,         % 自动换行(解决长行溢出)
    showspaces=false,        % 不显示空格符号
    showtabs=false,          % 不显示Tab符号
    tabsize=4,               % Tab缩进为4个字符
    escapeinside={/*@}{@*/}  % 可选:允许代码中嵌入LaTeX命令
}
% 加载图片处理包(支持png/jpg/pdf等格式)
\usepackage{graphicx}
% 可选:设置图片根目录(简化路径,推荐)
\graphicspath{{figures/}} % 表示所有图片默认从figures文件夹读取
\usepackage{tabularx}  % 自适应列宽,解决长文本溢出
\usepackage{booktabs}  % 美化表格横线(替代竖线,学术更规范)
\usepackage{siunitx}     % 数值小数位对齐(核心)
\usepackage{ragged2e}  % 优化单元格文本对齐
\usepackage{multirow}  % 因为用了\multirow
% 定义自适应列:左对齐+自动换行
\newcolumntype{Y}{>{\RaggedRight\arraybackslash}X}
% 定义数值列:保留2位小数,居中对齐
\newcolumntype{F}[1]{>{\centering\arraybackslash}S[table-format=#1, round-mode=places, round-precision=2]}

%%-------------------------------正文开始---------------------------%%
\begin{document}

%%-----------------------封面--------------------%%
\cover

%%------------------摘要-------------%%
%\begin{abstract}
%
%在此填写摘要内容
%
%\end{abstract}

\thispagestyle{empty} % 首页不显示页码

%%--------------------------目录页------------------------%%
\newpage
\tableofcontents

%%------------------------正文页从这里开始-------------------%
\newpage

%%可选择这里也放一个标题
%\begin{center}
%    \title{ \Huge \textbf{{标题}}}
%\end{center}

\section{引言:并行计算基础}

\subsection{并行计算概述}

并行计算是指同时使用多个计算资源解决计算问题的过程。随着处理器主频增长放缓,多核处理器成为提升计算性能的主要途径。通过将任务分解为可以并发执行的子任务,并行计算能够显著缩短程序运行时间,提高计算效率。

\subsubsection{并行计算的分类}

根据并行粒度的不同,并行计算可分为:

\begin{itemize}
    \item \textbf{任务级并行(Task-Level Parallelism)}:将不同的任务分配给不同的处理器执行
    \item \textbf{数据级并行(Data-Level Parallelism)}:对大规模数据集的不同部分同时进行相同操作
    \item \textbf{指令级并行(Instruction-Level Parallelism)}:在单个处理器内同时执行多条指令
\end{itemize}

本项目主要关注数据级并行,利用OpenMP实现共享内存多线程并行。

\subsection{OpenMP并行编程模型}

OpenMP(Open Multi-Processing)是一种支持共享内存并行编程的API,广泛应用于C/C++和Fortran程序中。其主要特点包括:

\begin{itemize}
    \item \textbf{Fork-Join模型}:主线程创建多个工作线程,并行执行后再汇合
    \item \textbf{编译器指令}:通过\texttt{\#pragma omp}指令控制并行行为
    \item \textbf{可移植性强}:支持多种编译器和操作系统
    \item \textbf{增量并行化}:可逐步对串行程序进行并行优化
\end{itemize}

常用的OpenMP并行指令包括:
\begin{itemize}
    \item \texttt{\#pragma omp parallel for}:并行执行for循环
    \item \texttt{\#pragma omp parallel sections}:并行执行不同的代码段
    \item \texttt{\#pragma omp critical}:保护临界区
    \item \texttt{\#pragma omp barrier}:设置同步点
\end{itemize}


\begin{figure}[!htbp]
    \centering
    \includegraphics[width =.8\textwidth]{fork-join.png}
    \caption{OpenMP Fork-Join模型示意图}
    \label{fig:openmp_fork_join}
\end{figure}

\subsection{性能评估指标}

\subsubsection{加速比(Speedup)}

加速比是衡量并行性能最常用的指标,定义为:

\begin{equation}
    S_p = \frac{T_1}{T_p}
    \label{eq:speedup}
\end{equation}

其中$T_1$为串行执行时间,$T_p$为使用$p$个处理器的并行执行时间。理想情况下$S_p = p$(线性加速),但实际中通常$S_p < p$。

\subsubsection{并行效率(Efficiency)}

并行效率衡量处理器资源的利用率:

\begin{equation}
    E_p = \frac{S_p}{p} = \frac{T_1}{p \cdot T_p}
    \label{eq:efficiency}
\end{equation}

理想情况下$E_p = 1$(100\%),表示所有处理器都被充分利用。

\subsubsection{Amdahl定律}

Amdahl定律描述了程序中串行部分对并行加速比的限制:

\begin{equation}
    S_p \leq \frac{1}{f + \frac{1-f}{p}}
    \label{eq:amdahl}
\end{equation}

其中$f$是程序中不可并行部分的比例。当$p \to \infty$时,$S_p \to \frac{1}{f}$。这说明即使有无限多处理器,加速比也受串行部分的限制。


\subsection{Roofline模型}

Roofline模型是分析程序性能的重要工具,它描述了性能上限由两个因素决定:

\begin{equation}
    \text{Performance} = \min(\text{Peak FLOPS}, \text{Arithmetic Intensity} \times \text{Memory Bandwidth})
    \label{eq:roofline}
\end{equation}



\comment{需要调整位置}
\begin{figure}[!htbp]
    \centering
    \includegraphics[width =.7\textwidth]{roofline.png}
    \caption{Roofline模型示意图}
    \label{fig:roofline_model}
\end{figure}
图中的 $I_{\text{max}}$ 为拐点所对应的强度,是计算平台的计算强度上限。当 $I < I_{\text{max}}$ 时,模型的计算性能受限于带宽大小,处于带宽瓶颈区;当 $I > I_{\text{max}}$ 时,模型的计算性能瓶颈取决于平台的算力,处于计算瓶颈区。所以,模型的计算强度应尽量大于 $I_{\text{max}}$,这样才能最大程度利用计算平台的算力资源,但超过之后就无需追求无意义的提升,因为再强性能都只能达到计算平台的算力。



\subsection{性能benchmark与测试环境}

本项目的所有性能测试均在以下环境下进行:

\begin{itemize}
    \item \textbf{处理器}:AMD Ryzen AI 9 H 365 w/ Radeon 880M(10核心20线程,基频2.0GHz)
    \item \textbf{内存}:32GB
    \item \textbf{编译器}:GCC 15.2.0 (MinGW),Clang 17.0.6
    \item \textbf{操作系统}:Windows 11 家庭中文版
    \item \textbf{并行库}:OpenMP
\end{itemize}

性能测试方法:
\begin{itemize}
    \item 测试算子时先50次warm up,再200次取中位数和P99
    \item 使用高精度计时器(\texttt{std::chrono::high\_resolution\_clock})
    \item 测试前进行warm-up避免冷启动影响
\end{itemize}



%\section{定理环境}
%\begin{Theorem}
%\end{Theorem}
%
%\begin{Lemma}
%\end{Lemma}
%
%\begin{Corollary}
%\end{Corollary}
%
%\begin{Proposition}
%\end{Proposition}
%
%\begin{Definition}
%\end{Definition}
%
%\begin{Example}
%\end{Example}
%
%\begin{proof}
%\end{proof}

% \subsection{插入参考文献}
% 直接使用\verb|\cite{}|即可。

% 例如:


%    \textit{ 此处引用了文献\cite{OFDMAbackscatter}。此处引用了文献\cite{DigiScatter}}


% 引用过的文献会自动出现在参考文献中。
\clearpage
\section{神经网络算子并行化}

神经网络中的算子计算通常具有高度的数据并行性,通过OpenMP等并行技术可以显著提升计算性能。本节介绍两个典型的神经网络算子:卷积(Convolution)和平均池化(Average Pooling)的并行实现。
\subsection{计算密集型 vs 访存密集型} 


在并行计算中,算子的性能特征可以分为两类:

\subsubsection{计算密集型(Compute-Bound)}

\textbf{定义:}算法的性能瓶颈在于计算单元(ALU、FPU)的吞吐量,而非内存带宽。

\textbf{特征:}
\begin{itemize}
    \item 高计算访存比(Compute-to-Memory Ratio):每次内存访问对应大量计算操作
    \item 算术强度大:浮点运算次数/内存访问字节数 $\gg$ 1
    \item 瓶颈:处理器的浮点运算能力
    \item 并行效果:通常能获得较好的加速比
\end{itemize}

\textbf{典型例子:}
\begin{itemize}
    \item \textbf{卷积操作}:以输入$(1,3,150,150)$、输出$(1,32,150,150)$、stride=1、padding=2、kernel\_size=$(5,5)$为例,算术强度约为386.27 FLOP/byte
    \item \textbf{矩阵乘法}:对于$n \times n$矩阵,算术强度约为$n/3$ FLOP/byte,$n$越大越显著
\end{itemize}

\textbf{优化策略:}增加并行度(多线程、SIMD)、提高指令级并行、循环展开

\subsubsection{访存密集型(Memory-Bound)}

\textbf{定义:}算法的性能瓶颈在于内存系统的带宽,而非计算能力。

\textbf{特征:}
\begin{itemize}
    \item 低计算访存比:每次内存访问对应很少的计算操作
    \item 算术强度小:浮点运算次数/内存访问字节数 $\ll$ 1
    \item 瓶颈:内存带宽、缓存命中率
    \item 并行效果:容易受内存带宽限制,加速比受限
\end{itemize}

\textbf{典型例子:}
\begin{itemize}
    \item \textbf{池化操作}:对于$2\times2$池化,算术强度约为0.31 FLOP/byte
    \item \textbf{Batch Normalization}:主要是内存读写和简单运算
    \item \textbf{激活函数}(ReLU等):逐元素操作,计算量极小
\end{itemize}

\textbf{优化策略:}提高缓存命中率(数据重用、分块)、减少内存访问次数、预取优化、内存访问模式优化
\comment{需要调整位置}
\begin{table}[!htbp]
    \centering
    \caption{计算密集型与访存密集型的并行化特征对比}
    \label{tab:compute_vs_memory}
    \begin{tabular}{c|c|c|c}
    \hline
        类型 & 并行加速比 & 主要挑战 & 优化重点 \\
        \hline
        计算密集型 & 接近线性 & 负载均衡 & 增加计算并行度 \\
        访存密集型 & 受限 & 内存带宽竞争 & 减少内存访问、提高缓存效率 \\
        \hline
    \end{tabular}
\end{table}
\begin{itemize}
    \item \textbf{计算密集型}:性能接近Peak FLOPS(屋顶的水平部分)
    \item \textbf{访存密集型}:性能受限于Memory Bandwidth(屋顶的斜线部分)
\end{itemize}
\subsection{卷积算子(Conv2d) - 计算密集型}

\subsubsection{算法原理}

二维卷积是深度学习中最基础且最重要的操作之一。给定输入特征图$X \in \mathbb{R}^{C_{in} \times H_{in} \times W_{in}}$和卷积核$W \in \mathbb{R}^{C_{out} \times C_{in} \times K_h \times K_w}$,卷积操作计算输出特征图$Y \in \mathbb{R}^{C_{out} \times H_{out} \times W_{out}}$:

\begin{equation}
    Y[o, h, w] = b[o] + \sum_{c=0}^{C_{in}-1} \sum_{i=0}^{K_h-1} \sum_{j=0}^{K_w-1} X[c, h \cdot s_h + i, w \cdot s_w + j] \cdot W[o, c, i, j]
    \label{eq:conv2d}
\end{equation}

其中$C_{in}$, $C_{out}$为输入和输出通道数,$K_h$, $K_w$为卷积核高度和宽度,$s_h$, $s_w$为步长,$b[o]$为偏置项。输出特征图的尺寸计算公式:

\begin{equation}
    H_{out} = \left\lfloor \frac{H_{in} + 2 \cdot \text{padding} - K_h}{s_h} \right\rfloor + 1
    \label{eq:conv_output_size}
\end{equation}

\subsubsection{实现策略演进}

本项目实现了三个版本的卷积算子,展示了从串行到并行、再到深度优化的演进过程。

\textbf{版本1:串行实现} - 使用动态计数器记录输出位置,逻辑简单但存在数据依赖,无法直接并行。
\begin{lstlisting}[language=C++]
// 动态计数器,记录每个通道的输出位置
int cnt[100];
memset(cnt, 0, sizeof cnt);
int weight_pos = 0;
for (int d = 0; d < padded_mat.dim; ++d) {  // batch维度
    for (int i = 0; i < output.channel; ++i) {  // 输出通道
        for (int c = 0; c < padded_mat.channel; ++c) {  // 输入通道
            for (int h = 0; h < padded_mat.height; h += conv_stride[0]) {
                if (h + conv_kernel_size[0] > padded_mat.height) continue;
                for (int w = 0; w < padded_mat.width; w += conv_stride[1]) {
                    if (w + conv_kernel_size[1] > padded_mat.width) continue;
                    int index = d * pc * ph * pw + c * ph * pw + h * pw + w;
                    sum = 0;
                    // 卷积核计算
                    for (int k = 0; k < conv_kernel_max; ++k) {
                        sum += (padded_mat[index + dx[k]] * weight[weight_pos + k]);
                    }
                    output[cnt[c]++] += sum;  // 动态累加到输出
                }
            }
            weight_pos += conv_kernel_max;  // 权重位置累加
        }
    }
}
\end{lstlisting}
\textbf{问题分析:}
\begin{itemize}
    \item \texttt{cnt[c]++} 存在数据竞争,无法直接并行化;
    \item \texttt{weight\_pos} 采用循环累加方式,存在跨线程依赖;
    \item 索引计算逻辑复杂,显著降低执行性能。
\end{itemize}
\textbf{版本2:并行实现} - 改为静态索引计算,在输出通道维度并行,消除依赖但并行粒度仅32。

\begin{lstlisting}[language=C++]
    #pragma omp parallel for  // 在输出通道维度并行
    for (int i = 0; i < output.channel; ++i) {
        for (int d = 0; d < padded_mat.dim; ++d) {
            for (int c = 0; c < padded_mat.channel; ++c) {
                // 静态计算权重位置,消除依赖
                int weight_pos = i * padded_mat.channel * conv_kernel_max + c * conv_kernel_max;
                for (int h = 0; h < input.height; h += conv_stride[0]) {
                    for (int w = 0; w < input.width; w += conv_stride[1]) {
                        // 静态计算输出位置
                        int index = d * padded_mat.channel * padded_mat.height * padded_mat.width 
                                + c * padded_mat.height * padded_mat.width + h * padded_mat.width + w;
                        int output_index = i * output.height * output.width + h * output.width + w;
                        // 卷积核计算
                        for (int m = 0; m < conv_kernel_max; ++m) {
                            output[output_index] += (padded_mat[index + dx[m]] * weight[weight_pos + m]);
                        }
                    }
                }
            }
        }
    }
\end{lstlisting}

\comment{格式还要改}
\textbf{问题分析:}
\begin{itemize}
    \item 索引计算仍然复杂,计算开销大;
    \item 多次重复计算相同的索引;
    \item 可能存在伪共享(False Sharing)问题。
\end{itemize}

\textbf{版本3:深度优化实现} - 针对版本2的瓶颈实现三方面优化:
\begin{enumerate}
    \item \textbf{并行化Padding}:按通道并行+\texttt{memcpy}批量复制,padding时间降至原来的1/5
    \item \textbf{二维空间并行}:使用\texttt{collapse(2)}合并$h \times w$循环,任务数从32提升至22,500,负载均衡显著改善
    \item \textbf{Bias融合}:在卷积计算时直接加bias,消除720K次额外内存访问
\end{enumerate}
\begin{lstlisting}[language=C++]
    // 二维空间并行:任务数 = 150 × 150 = 22,500
    // 每个线程处理一个输出位置的所有通道,避免false sharing
    #pragma omp parallel for collapse(2) schedule(static)
    for (int oh = 0; oh < out_h; ++oh) {
        for (int ow = 0; ow < out_w; ++ow) {
            int h_start = oh * stride_h;
            int w_start = ow * stride_w;
            
            // 每个输出点独立计算所有输出通道
            for (int oc = 0; oc < channel_out; ++oc) {
                float sum = 0.0f;
                
                // 遍历所有输入通道
                for (int ic = 0; ic < channel_in; ++ic) {
                    const float* input_ptr = &padded_mat.tensor[
                        ic * in_h * in_w + h_start * in_w + w_start];
                    const float* weight_ptr = &weight[oc * channel_in * kernel_max + ic * kernel_max];
                    
                    // 5×5卷积核完全手动展开(25项)
                    ...(展开代码略)
                }
                
                // 优化:直接加bias再写入,避免后续额外遍历
                output.tensor[oc * out_h * out_w + oh * out_w + ow] = sum + bias[oc];
            }
        }
    }
\end{lstlisting}
如\autoref{tab:conv_version_comparison}所示,三个版本的性能对比清晰展现了优化效果。

\begin{table}[htbp]
    \centering
    \caption{普通并行 vs 优化后并行性能对比}
    \label{tab:parallel_perf}
    \resizebox{\textwidth}{!}{
    \begin{tabular}{c|ccc|ccc|c}
        \toprule
        \multirow{2}{*}{\textbf{线程数}} 
        & \multicolumn{3}{c|}{\textbf{普通并行}} 
        & \multicolumn{3}{c|}{\textbf{优化后并行}} 
        & \multirow{2}{*}{\textbf{性能提升率}} \\
        \cmidrule(lr){2-4} \cmidrule(lr){5-7}
        & \textbf{中位数时间(ms)} 
        & \textbf{加速比} 
        & \textbf{效率(\%)} 
        & \textbf{中位数时间(ms)} 
        & \textbf{加速比} 
        & \textbf{效率(\%)} 
        & \textbf{(vs 普通并行)} \\
        \midrule
        1  & 18.20 & 1.00 & 100.00 & 18.58 & 1.00 & 100.00 & -2.1\% \\
        2  & 9.88  & 1.84 & 91.90  & 9.32  & 1.99 & 99.50  & +5.7\% \\
        4  & 5.52  & 3.30 & 82.54  & 4.93  & 3.77 & 94.25  & +10.7\% \\
        8  & 4.01  & 4.54 & 56.72  & 3.72  & 5.00 & 62.50  & +7.2\% \\
        10 & 4.05  & 4.49 & 44.91  & 2.82  & 6.59 & 65.90  & +30.4\% \\
        16 & 3.63  & 5.01 & 31.32  & 2.65  & 7.00 & 43.75  & +27.0\% \\
        20 & 3.54  & 5.14 & 25.68  & 2.36  & 7.86 & 39.30  & \textbf{+33.3\%} \\
        \bottomrule
    \end{tabular}
    }
\end{table}

\subsubsection{并行化挑战与优化}

\textbf{1. 并行粒度选择} - 如\autoref{tab:conv_parallel_granularity}所示,不同并行维度性能差异巨大。

\begin{table}[!htbp]
    \centering
    \caption{卷积不同并行粒度的性能对比}
    \label{tab:conv_parallel_granularity}
    \begin{tabular}{c|c|c|c|c}
    \hline
        并行维度 & 并行数 & 每线程计算量 & 开销占比 & 推荐度 \\
        \hline
        width(w) & 150 & $\sim$600 FLOP & >99\% & ❌ 极差 \\
        height(h) & 150 & $\sim$90K FLOP & $\sim$10\% & ⚠️ 较差 \\
        输入通道(ic) & 3 & $\sim$16M FLOP & <0.1\% & ⚠️ 一般 \\
        \textbf{输出通道(oc)} & \textbf{32} & \textbf{$\sim$1.7M FLOP} & \textbf{<1\%} & \textbf{✅ 最佳} \\
        \hline
    \end{tabular}
\end{table}

\textbf{关键发现:}width并行灾难性失败(20线程反而慢69×),原因是线程开销远大于计算时间;输出通道并行表现最佳;二维空间并行进一步优化(加速7.86×)。

\textbf{2. False Sharing} - 解决方案:按输出通道并行或二维空间并行,每线程写入独立位置。

\textbf{3. 内存带宽限制} - 算术强度约386.27 FLOP/byte,虽属计算密集型,但多线程时仍可能受带宽限制。这是由于本次测试用的张量维数较低,计算量不够大,对于现代CPU来说,瓶颈仍为内存访问。

\subsection{平均池化算子(AvgPool2d) - 访存密集型}

\subsubsection{算法原理}

平均池化对输入特征图的局部区域求平均,用于降采样。给定输入$X \in \mathbb{R}^{C \times H \times W}$和池化窗口$(K_h, K_w)$,输出$Y \in \mathbb{R}^{C \times H' \times W'}$:

\begin{equation}
    Y[c, h, w] = \frac{1}{K_h \times K_w} \sum_{i=0}^{K_h-1} \sum_{j=0}^{K_w-1} X[c, h \cdot s_h + i, w \cdot s_w + j]
    \label{eq:avgpool}
\end{equation}

\subsubsection{并行实现与性能}
\textbf{版本1:普通并行} - 仅添加\texttt{\#pragma omp parallel for},在通道维度并行。
\comment{括号可以放到同一行}
\begin{lstlisting}[language=C++]
    for (int d = 0; d < input.dim; ++d)
    {
        // #pragma omp parallel for (普通并行只要在这加入并行即可)
        for (int c = 0; c < input.channel; ++c)
        {
            for (int oh = 0; oh < out_h; ++oh)
            {
                for (int ow = 0; ow < out_w; ++ow)
                {
                    float sum = 0.0;
                    int count = 0;
                    for (int kh = 0; kh < avgp_kernel_size[0]; ++kh)
                    {
                        for (int kw = 0; kw < avgp_kernel_size[1]; ++kw)
                        {
                            int h = oh * avgp_stride[1] + kh;
                            int w = ow * avgp_stride[0] + kw;
                            if (h < input_h && w < input_w)
                            {
                                int index = (d * input.channel * input_h * input_w) + (c * input_h * input_w) + (h * input_w) + w;
                                sum += input[index];
                                count++;
                            }
                        }
                    }
                    output[(d * output.channel * out_h * out_w) + (c * out_h * out_w) + (oh * out_w) + ow] = sum / count;
                }
            }
        }
    }
\end{lstlisting}

\textbf{版本2:访存优化} - 通过预计算通道指针、优化边界检查和针对2x2池化的特殊优化,提升内存访问效率。

\begin{lstlisting}[language=C++]
    for (int d = 0; d < input.dim; ++d)
    {
        // Parallelize over channels - optimal granularity
        #pragma omp parallel for
        for (int c = 0; c < input.channel; ++c)
        {
            // Precompute channel offsets using pointers
            const float* input_channel_ptr = &input.tensor[d * input.channel * input_hw + c * input_hw];
            float* output_channel_ptr = &output.tensor[d * output.channel * output_hw + c * output_hw];
            
            if (use_2x2_optimized)
            {
                // Specialized optimization for 2x2 kernel with stride 2
                // Unroll kernel loops for better performance
                for (int oh = 0; oh < out_h; ++oh)
                {
                    for (int ow = 0; ow < out_w; ++ow)
                    {
                        // Input coordinates
                        int h_start = oh * 2;
                        int w_start = ow * 2;
                        
                        // Check if all 4 pixels are within bounds
                        if (h_start + 1 < input_h && w_start + 1 < input_w)
                        {
                            // Direct access to 4 pixels in 2x2 kernel
                            int idx_00 = h_start * input_w + w_start;
                            int idx_01 = idx_00 + 1;
                            int idx_10 = idx_00 + input_w;
                            int idx_11 = idx_10 + 1;
                            
                            // Compute average of 4 pixels
                            float sum = input_channel_ptr[idx_00] + 
                                       input_channel_ptr[idx_01] + 
                                       input_channel_ptr[idx_10] + 
                                       input_channel_ptr[idx_11];
                            
                            output_channel_ptr[oh * out_w + ow] = sum * 0.25f; // Multiply instead of divide
                        }
                        else
                        {
                            // Boundary case - use general approach
                            float sum = 0.0f;
                            int count = 0;
                            for (int kh = 0; kh < 2; ++kh)
                            {
                                for (int kw = 0; kw < 2; ++kw)
                                {
                                    int h = h_start + kh;
                                    int w = w_start + kw;
                                    if (h < input_h && w < input_w)
                                    {
                                        sum += input_channel_ptr[h * input_w + w];
                                        count++;
                                    }
                                }
                            }
                            output_channel_ptr[oh * out_w + ow] = sum / count;
                        }
                    }
                }
            }
            else
            {
                // General case for arbitrary kernel sizes
                for (int oh = 0; oh < out_h; ++oh)
                {
                    for (int ow = 0; ow < out_w; ++ow)
                    {
                        float sum = 0.0f;
                        int count = 0;
                        
                        int h_start = oh * stride_h;
                        int w_start = ow * stride_w;
                        
                        for (int kh = 0; kh < kernel_h; ++kh)
                        {
                            for (int kw = 0; kw < kernel_w; ++kw)
                            {
                                int h = h_start + kh;
                                int w = w_start + kw;
                                if (h < input_h && w < input_w)
                                {
                                    sum += input_channel_ptr[h * input_w + w];
                                    count++;
                                }
                            }
                        }
                        output_channel_ptr[oh * out_w + ow] = sum / count;
                    }
                }
            }
        }
    }
\end{lstlisting}

池化操作属典型访存密集型算子,在通道维度并行,使用连续内存访问模式提高缓存命中率。


\begin{table}[htbp]  % 浮动体位置参数,适配前文讲解的htbp
    \centering
    \caption{两种池化实现对比}  % 表格标题(按需修改)
    \label{tab:pool_optim}     % 交叉引用标签(按需修改)
    \resizebox{\linewidth}{!}{% 自适应页面宽度
        \begin{tabularx}{\linewidth}{cYYc}
            \toprule  % 顶部粗线
            \textbf{优化维度} & \textbf{普通版本} & \textbf{访存优化版本} & \textbf{性能影响} \\
            \midrule  % 中间细线
            \textbf{指针运算} & 每次循环重复计算全局索引 \texttt{(d*C*H*W + c*H*W + h*W + w)} & \textbf{预计算通道指针} \texttt{input\_channel\_ptr} 和 \texttt{output\_channel\_ptr} & \textbf{高} 消除大量乘法运算 \\
            \midrule
            \textbf{边界检查} & 内层循环逐像素\texttt{if(h < input\_h \&\& w < input\_w)} & \textbf{外提边界判断},内层循环无分支 & \textbf{高} 减少分支预测失败 \\
            \midrule
            \textbf{特殊Case特化} & 通用循环处理所有kernel size & \textbf{2×2 kernel stride=2 手动展开} & \textbf{极高} 向量化+指令级并行 \\
            \midrule
            \textbf{除法优化} & \texttt{sum / count} 浮点除法 & 2×2 case用 \texttt{* 0.25f} 乘法代替 & \textbf{中} 除法耗时是乘法的5-10倍 \\
            \midrule
            \textbf{访存连续性} & 跨通道访问导致跳跃 & \textbf{通道内连续访存} + 局部性优化 & \textbf{高} 提升Cache命中率 \\
            \bottomrule  % 底部粗线
        \end{tabularx}
    }
\end{table}
\begin{table}[!htbp]
    \centering
    \caption{平均池化算子性能测试结果}
    \label{tab:avgpool_performance}
    \resizebox{\textwidth}{!}{
    \begin{tabular}{l|c|c|c|c|c}
        \toprule
        \textbf{测试类型} & \textbf{线程数} & \textbf{中位数(ms)} & \textbf{P99(ms)} & \textbf{加速比} & \textbf{并行效率} \\
        \midrule
        \multirow{7}{*}{\textbf{input channel并行}} 
        & 1  & 31.12 & 34.25 & 1.00× & 100.00\% \\
        & 2  & 18.26 & 21.47 & 1.70× & 85.00\% \\
        & 4  & 10.14 & 13.59 & 3.07× & 76.75\% \\
        & 8  & 6.31  & 7.48  & 4.93× & 61.63\% \\
        & 10 & 5.99  & 7.28  & 5.20× & 52.00\% \\
        & 16 & 5.22  & 8.48  & 5.96× & 37.25\% \\
        & 20 & 4.97  & 6.81  & \textbf{6.26×} & 31.30\% \\
        \midrule
        \multirow{7}{*}{\textbf{访存优化版avgpool}} 
        & 1  & 12.70 & 14.29 & 1.00× & 100.00\% \\
        & 2  & 6.91  & 7.35  & 1.84× & 92.00\% \\
        & 4  & 4.20  & 5.05  & 3.02× & 75.50\% \\
        & 8  & 2.82  & 3.49  & 4.50× & 56.25\% \\
        & 10 & 2.57  & 3.32  & 4.94× & 49.40\% \\
        & 16 & 2.50  & 3.12  & 5.08× & 31.75\% \\
        & 20 & 2.66  & 3.11  & \textbf{4.77×} & 23.85\% \\
        \bottomrule
    \end{tabular}
    }
\end{table}

\textbf{关键发现:}池化加速比(4.45×@20线程)明显低于卷积(7.86×),并行效率下降更快,原因是算术强度低(0.31 FLOP/byte),受内存带宽限制。

\subsection{小结}

\begin{itemize}
    \item \textbf{计算密集型}(卷积):重点是合适的并行粒度、消除False Sharing、负载均衡
    \item \textbf{访存密集型}(池化):受内存带宽限制,需缓存优化、数据预取  
    \item \textbf{共同原则}:避免过细并行粒度、静态索引消除依赖、合理选择并行维度
\end{itemize}

\clearpage

\section{红黑排序Gauss-Seidel迭代法}

\subsection{算法原理}

Gauss-Seidel方法是求解线性方程组$Ax = b$的经典迭代方法,常用于求解泊松方程等偏微分方程的离散化系统。对于二维泊松方程:

$$
-\nabla^2 u = f, \quad (x, y) \in \Omega
$$

使用有限差分离散化后得到五点模板:

$$
\frac{u_{i-1,j} + u_{i+1,j} + u_{i,j-1} + u_{i,j+1} - 4u_{i,j}}{h^2} = f_{i,j}
$$

标准Gauss-Seidel迭代格式:

$$
u_{i,j}^{(k+1)} = \frac{1}{4}(u_{i-1,j}^{(k+1)} + u_{i+1,j}^{(k)} + u_{i,j-1}^{(k+1)} + u_{i,j+1}^{(k)} - h^2 f_{i,j})
$$

\textbf{数据依赖问题:}标准Gauss-Seidel方法在更新$u_{i,j}$时依赖于已更新的$u_{i-1,j}$和$u_{i,j-1}$,这种\textbf{读后写(RAW, Read-After-Write)依赖}使得算法难以直接并行化。

\subsection{红黑排序(Red-Black Ordering)}

红黑排序通过将网格点按照\textbf{棋盘染色}方式分为两类,打破了串行数据依赖:

\begin{itemize}
    \item \textbf{二维情况}:红点($i + j$为偶数),黑点($i + j$为奇数)
    \item \textbf{三维情况}:红点($i + j + k$为偶数),黑点($i + j + k$为奇数)
\end{itemize}

\textbf{关键性质}(打破依赖的核心):
\begin{itemize}
    \item 红点的四邻点(2D)或六邻点(3D)\textbf{全部是黑点}
    \item 黑点的邻点\textbf{全部是红点}
    \item 因此:\textbf{同色点之间无依赖},可以完全并行更新
\end{itemize}

\begin{figure}
    \centering
    \includegraphics[width=0.4\textwidth]{figures/RB-gauss.png}
    \caption{二维红黑网格示意图}
    \label{fig:red_black_grid}
\end{figure}

迭代步骤:
\begin{enumerate}
    \item \textbf{红点更新阶段}(并行):$u_{red}^{(k+1)} = f(\text{黑色邻点}^{(k)})$
    \item \textbf{屏障同步}:确保所有红点更新完成
    \item \textbf{黑点更新阶段}(并行):$u_{black}^{(k+1)} = f(\text{红色邻点}^{(k+1)})$  
    \item \textbf{屏障同步}:确保所有黑点更新完成
\end{enumerate}

\subsection{OpenMP并行化与Barrier同步机制}

\subsubsection{Barrier同步原理}

\textbf{Barrier(屏障)}是一种线程同步原语,确保所有线程到达同一执行点后才能继续。在OpenMP中:

\begin{itemize}
    \item \texttt{\#pragma omp for}:循环结束时有\textbf{隐式barrier}
    \item \texttt{\#pragma omp for nowait}:取消隐式barrier
    \item \texttt{\#pragma omp barrier}:显式同步点
\end{itemize}

\subsubsection{红黑GS中的Barrier需求}

红黑GS每次迭代需要2个Barrier:

\begin{itemize}
    \item \textbf{Barrier \#1}:确保所有红点更新完成后,黑点才能读取新值
    \item \textbf{Barrier \#2}:确保所有黑点更新完成后,下一次迭代才能开始
\end{itemize}

\textbf{必须同步的原因:}
\begin{itemize}
    \item 数据依赖:黑点更新需要\textbf{所有}红点的新值(跨线程依赖)
    \item 内存一致性:多核CPU的缓存需要同步
    \item 迭代正确性:第$k+1$次迭代必须基于第$k$次迭代的完整结果
\end{itemize}

\subsection{小规模问题的并行困境}

\subsubsection{为什么小规模问题"越并行越慢"?}

如\autoref{tab:gs_small_scale}所示,小规模问题出现严重的并行负优化现象。

\begin{table}[!htbp]
    \centering
    \caption{小规模问题并行性能(64×64网格,10000次迭代)}
    \label{tab:gs_small_scale}
    \begin{tabular}{c|c|c|c}
    \hline
        线程数 & 执行时间(ms) & vs单线程 & 同步开销占比 \\
        \hline
        1 & 20.74 & 1.00× & 0\% \\
        2 & 288.37 & 0.07×  & $\sim$92\% \\
        4 & 653.93 & 0.03×  & $\sim$97\% \\
        8 & 1209.27 & 0.02×  & $\sim$98\% \\
        \hline
    \end{tabular}
\end{table}

\textbf{现象:}2线程比单线程慢14倍,8线程慢58倍!

\textbf{根本原因分析:}

\textbf{① 计算量极小,远小于同步开销}

\begin{itemize}
    \item 计算量(单次迭代,64×64网格):2048个点 × 6 FLOPs = 12,288 FLOPs
    \item 单核计算时间(假设3.5 GHz,4 FLOP/cycle):$\approx$ 0.88微秒
    \item 同步开销(单次Barrier):8线程barrier延迟 $\approx$ 5-20微秒
    \item 每次迭代2个barrier:10-40微秒
\end{itemize}

\textbf{结论:同步时间 >> 计算时间(10倍以上)!}

\textbf{② Amdahl定律的体现}

对于小规模问题(64×64,10000次迭代):
\begin{itemize}
    \item 计算时间:0.88μs × 10000 × 2 = 17.6 ms(可并行)
    \item 同步时间:15μs × 10000 × 2 = 300 ms(串行)
    \item 串行占比:$1-P = \frac{300}{300+17.6} \approx 94\%$
\end{itemize}

代入Amdahl定律(8线程):

$$
S_p = \frac{1}{0.94 + \frac{0.06}{8}} \approx 1.06\times
$$

\textbf{实际结果:0.02×(远低于理论值)!}原因包括:缓存失效、NUMA远程内存访问、False Sharing、操作系统调度抖动。

\subsubsection{规模阈值分析}

根据经验法则,要求$\frac{T_{compute}}{T_{sync}} > 10$。推导规模下限(2D情况):

设网格规模为$N \times N$,要求:

$$
\frac{3N^2 / \text{Throughput}}{2 \times T_{barrier}} > 10
$$

代入参数(3.5 GHz,4 FLOP/cycle,$T_{barrier}=15\mu s$),解得:$N > 1183$

\textbf{结论:对于2D问题,$N < 1024$时并行收益有限!}

实测验证如\autoref{tab:gs_scale_threshold}所示。

\begin{table}[!htbp]
    \centering
    \caption{2D问题性能分界点(Original方法)}
    \label{tab:gs_scale_threshold}
    \begin{tabular}{c|c|c|c|c}
    \hline
        网格规模 & 2线程 & 4线程 & 8线程 & 结论 \\
        \hline
        64×64 & 0.06× ❌ & 0.03× ❌ & 0.02× ❌ & 完全失败 \\
        128×128 & 0.26× ❌ & 0.13× ❌ & 0.05× ❌ & 灾难性 \\
        256×256 & 0.79× ⚠️ & 0.62× ❌ & 0.27× ❌ & 仍然负优化 \\
        512×512 & 1.66× ✅ & 1.89× ✅ & 1.71× ✅ & \textbf{开始有效} \\
        1024×1024 & 1.89× ✅ & 2.58× ✅ & 3.30× ✅ & 效果良好 \\
        \hline
    \end{tabular}
\end{table}

\textbf{关键发现:}
\begin{itemize}
    \item 2D临界点:512×512规模开始有效并行(与理论预测$N>1183$基本吻合)
    \item 3D优势明显:由于计算量是2D的$N$倍,64³就能获得有效加速
    \item 小规模灾难:64×64和128×128的2D问题,多线程反而慢5-50倍
\end{itemize}

\subsection{两种优化方法}

本节介绍两种不同的Gauss-Seidel并行实现方法,从基础的红黑排序并行到高级的分块优化。

\subsubsection{Original方法:基础红黑排序并行}

\textbf{核心思想:}直接并行化红黑点更新

\textbf{关键设计点:}
\begin{itemize}
    \item \textbf{消除条件分支}:使用\texttt{j += 2}步长直接访问同色点,避免\texttt{if ((i+j) \% 2 == 0)}判断
    \item \textbf{自适应块大小}:小规模用小块(16-32),大规模用大块(64-128),平衡并行度和缓存
    \item \textbf{静态调度}:\texttt{schedule(static)}确保负载均匀分配
\end{itemize}

\begin{lstlisting}[language=C++]
    // gauss_seidel_2d.cpp 核心代码
void solve_parallel_redblack(/* ... */) {
    omp_set_num_threads(num_threads);
    
    // 自适应块大小(根据规模和线程数)
    int tile_size;
    if (N <= 64) {
        tile_size = (num_threads >= 4) ? 16 : 32;
    } else if (N <= 128) {
        tile_size = (num_threads >= 8) ? 16 : 32;
    } else if (N <= 256) {
        tile_size = 32;
    } else if (N <= 512) {
        tile_size = 64;
    } else {
        tile_size = 128;
    }
    
    for (int iter = 0; iter < max_iter; ++iter) {
        // === 红点更新阶段 ===
        #pragma omp for schedule(static) collapse(2) nowait
        for (int bi = 1; bi <= N; bi += tile_size) {
            for (int bj = 1; bj <= N; bj += tile_size) {
                // 块内更新红点
                for (int i = bi; i < i_end; ++i) {
                    // 关键优化:直接步长为2,无条件判断
                    int j_start = ((i + bi) % 2 == 0) ? bi : bi + 1;
                    for (int j = j_start; j < j_end; j += 2) {
                        U(i, j) = 0.25 * (U(i-1,j) + U(i+1,j) + 
                                         U(i,j-1) + U(i,j+1) + h2 * F(i-1,j-1));
                    }
                }
            }
        }
        #pragma omp barrier  // 确保所有红点更新完成
        // === 黑点更新阶段 ===
        #pragma omp for schedule(static) collapse(2) nowait
        for (int bi = 1; bi <= N; bi += tile_size) {
            for (int bj = 1; bj <= N; bj += tile_size) {
                // 块内更新黑点(步长同样为2)
                for (int i = bi; i < i_end; ++i) {
                    int j_start = ((i + bi) % 2 == 1) ? bi : bi + 1;
                    for (int j = j_start; j < j_end; j += 2) {
                        U(i, j) = 0.25 * (U(i-1,j) + U(i+1,j) + 
                                         U(i,j-1) + U(i,j+1) + h2 * F(i-1,j-1));
                    }
                }
            }
        }
        
        #pragma omp barrier  // 确保所有黑点更新完成
    }
}
\end{lstlisting}

\textbf{优点:}实现简单,对大规模问题(512²以上)效果良好

\textbf{缺点:}小规模问题(<512²)同步开销占主导,严重负优化;缓存利用率不理想

\subsubsection{Tiled方法:多级缓存分块优化}

\textbf{核心思想:}针对L1/L3 Cache的两级分块策略

\textbf{优化技术:}
\begin{itemize}
    \item \textbf{两级分块}:外层L3块(128×128),内层L1块(16×16)
    \item \textbf{寄存器缓存}:显式将邻点值缓存到寄存器,减少重复内存访问
    \item \textbf{预取优化}:提前加载下一个L1块(x86平台)
\end{itemize}

\textbf{性能提升:}在中大规模问题中稳定领先Original方法10-20\%
\begin{lstlisting}[language=C++]
    // gauss_seidel_2d_tiled.cpp 核心代码
void solve_4level_tiling(/* ... */) {
    // 两级tiling参数
    const int L3_TILE = (N >= 512) ? 128 : 64;  // 外层:L3 Cache块
    const int L1_TILE = 16;                      // 内层:L1 Cache块
    
    for (int iter = 0; iter < max_iter; ++iter) {
        // === 红点更新:两级分块 ===
        #pragma omp for schedule(static) nowait
        for (int bi = 1; bi <= N; bi += L3_TILE) {
            int bi_end = std::min(bi + L3_TILE, N + 1);
            
            // L1 Cache级别的细粒度分块
            for (int ti = bi; ti < bi_end; ti += L1_TILE) {
                int ti_end = std::min(ti + L1_TILE, bi_end);
                
                // 预取优化:提前加载下一个L1块
                #ifdef __x86_64__
                if (ti + L1_TILE < bi_end) {
                    _mm_prefetch((const char*)&U(ti + L1_TILE, 1), _MM_HINT_T0);
                }
                #endif
                
                // 内核循环:访问L1块内的红点
                for (int i = ti; i < ti_end; ++i) {
                    int j_start = (i % 2 == 1) ? 1 : 2;  // 红点起始位置
                    
                    for (int j = j_start; j <= N; j += 2) {
                        // 显式寄存器缓存邻点值
                        double reg[4];
                        reg[0] = U(i-1, j);
                        reg[1] = U(i+1, j);
                        reg[2] = U(i, j-1);
                        reg[3] = U(i, j+1);
                        
                        U(i, j) = 0.25 * (reg[0] + reg[1] + reg[2] + reg[3] + 
                                         h2 * F(i-1, j-1));
                    }
                }
            }
        }
        
        #pragma omp barrier
        
        // === 黑点更新:相同的两级分块策略 ===
        // ... (结构相同)
    }
}
\end{lstlisting}


\subsection{性能测试结果}

详细的性能测试数据见附录。我们只对比具有实际并行收益(加速比>1)的规模,主要发现如\autoref{tab:gs_methods_summary}所示。

\begin{table}[!htbp]
    \centering
    \caption{Original vs Tiled性能对比(最优配置下的绝对时间)}
    \label{tab:gs_methods_summary}
    \begin{tabular}{c|c|c|c}
    \hline
        问题类型 & 优胜方法 & 最优时间 & 性能提升 \\
        \hline
        2D, 512×512 & \textbf{Tiled} & 86.86ms (4线程) & 快27\% \\
        2D, 1024×1024 & \textbf{Tiled} & 272.70ms (8线程) & \textbf{快43\%} \\
        2D, 2048×2048 & \textbf{Tiled} & 2810.09ms (8线程) & 快15\% \\
        \hline
        3D, 64³ & \textbf{Tiled} & 15.88ms (4线程) & 快14\% \\
        3D, 128³ & \textbf{Tiled} & 228.51ms (8线程) & 快26\% \\
        3D, 256³ & Original & 2374.13ms (8线程) & 快7\% \\
        3D, 512³ & \textbf{Tiled} & 16941.90ms (16线程) & 快3\% \\
        \hline
    \end{tabular}
\end{table}

\textbf{关键发现:}
\begin{itemize}
    \item \textbf{Tiled全面领先}:在7个测试规模中,Tiled方法赢得6场,仅在3D 256³输给Original
    \item \textbf{中等规模优势显著}:1024×1024(快43\%)和128³(快26\%)规模下,分块优化效果最佳
    \item \textbf{超大规模仍有效}:2048×2048和512³规模下,Tiled依然保持15\%和3\%的领先
    \item \textbf{唯一例外}:3D 256³规模下Original快7\%,可能因缓存容量与分块大小不匹配
\end{itemize}

\subsection{实用性建议}

\textbf{方法选择决策树:}

\begin{itemize}
    \item \textbf{2D问题:}
    \begin{itemize}
        \item $N < 512$:使用\textbf{串行版本}(并行完全无收益)
        \item $N \geq 512$:推荐\textbf{Tiled方法}(全面领先15-43\%)
        \begin{itemize}
            \item 512×512: 4线程(86.86ms)
            \item 1024×1024: 8线程(272.70ms,优势最大)
            \item 2048×2048: 8线程(2810.09ms)
        \end{itemize}
    \end{itemize}
    
    \item \textbf{3D问题:}
    \begin{itemize}
        \item $N < 64$:使用\textbf{串行版本}
        \item $N = 256$:推荐\textbf{Original,8线程}(唯一例外,快7\%)
        \item 其他规模:推荐\textbf{Tiled方法}
        \begin{itemize}
            \item 64³: 4线程(15.88ms,快14\%)
            \item 128³: 8线程(228.51ms,快26\%)
            \item 512³: 16线程(16941.90ms)
        \end{itemize}
    \end{itemize}
\end{itemize}

\textbf{线程配置建议:}
\begin{itemize}
    \item ✓ 中等规模(2D: 512-1024,3D: 64-128):4-8线程最优
    \item ✓ 超大规模(2D: 2048+,3D: 512+):8-16线程
    \item ❌ 避免超过16线程:实测20线程性能反而下降5-15\%
    \item ❌ 小规模禁用并行:同步开销占比>90\%,严重负优化
\end{itemize}

\subsection{小结}

本节通过红黑排序Gauss-Seidel迭代法的并行化实现,揭示了迭代算法并行化的核心挑战:

\begin{itemize}
    \item \textbf{打破数据依赖}:红黑排序成功打破了串行依赖,使并行化成为可能
    \item \textbf{同步开销主导}:小规模问题中,Barrier同步时间远大于计算时间(占比>90\%),导致并行完全无效
    \item \textbf{规模阈值效应}:存在明确的并行收益阈值——2D需$N \geq 512$,3D需$N \geq 64$
    \item \textbf{分块优化显著有效}:Tiled在7个测试规模中赢得6场,中等规模提升15-43\%
    \item \textbf{最佳配置实测}:1024×1024规模下,Tiled+8线程达到272.70ms(比Original快43\%,相对单线程3.63×加速)
\end{itemize}

\section{三对角方程组并行求解}

\subsection{问题定义}

三对角方程组是指系数矩阵只有主对角线及其上下相邻对角线非零的线性方程组:

$$
\begin{bmatrix}
b_0 & c_0 & & & \\
a_1 & b_1 & c_1 & & \\
& a_2 & b_2 & c_2 & \\
& & \ddots & \ddots & \ddots \\
& & & a_{n-1} & b_{n-1}
\end{bmatrix}
\begin{bmatrix}
x_0 \\ x_1 \\ x_2 \\ \vdots \\ x_{n-1}
\end{bmatrix}
=
\begin{bmatrix}
d_0 \\ d_1 \\ d_2 \\ \vdots \\ d_{n-1}
\end{bmatrix}
$$

三对角方程组广泛出现于:
\begin{itemize}
    \item 一维热传导方程的隐式差分格式
    \item 样条插值问题
    \item 边界值问题的数值解
    \item 金融工程中的期权定价模型
\end{itemize}

\subsection{Thomas算法(串行基准)}

Thomas算法(又称追赶法)是求解三对角方程组的经典串行算法,时间复杂度$O(n)$。

\textbf{算法步骤:}

\textbf{1. 前向消元(Forward Elimination):}

$$
\gamma_i = \begin{cases}
\frac{c_0}{b_0}, & i = 0 \\
\frac{c_i}{b_i - a_i \gamma_{i-1}}, & i = 1, 2, \ldots, n-2
\end{cases}
$$

$$
\rho_i = \begin{cases}
\frac{d_0}{b_0}, & i = 0 \\
\frac{d_i - a_i \rho_{i-1}}{b_i - a_i \gamma_{i-1}}, & i = 1, 2, \ldots, n-1
\end{cases}
$$

\textbf{2. 回代(Back Substitution):}

$$
x_i = \begin{cases}
\rho_{n-1}, & i = n-1 \\
\rho_i - \gamma_i x_{i+1}, & i = n-2, n-3, \ldots, 0
\end{cases}
$$

\textbf{数据依赖性分析:}

Thomas算法存在严重的\textbf{串行数据依赖}:
\begin{itemize}
    \item \textbf{前向消元}:第$i$步依赖第$i-1$步的$\gamma_{i-1}$和$\rho_{i-1}$
    \item \textbf{回代}:第$i$步依赖第$i+1$步的$x_{i+1}$
\end{itemize}

这种依赖链使得算法\textbf{无法直接并行化},必须采用特殊的并行策略。

\subsection{Brugnano并行算法 - 区域分解法}

Brugnano方法通过\textbf{区域分解(Domain Decomposition)}实现并行化,核心思想是将原问题分解为多个独立的子问题。

\subsubsection{算法原理}

\textbf{四个关键步骤:}

\textbf{步骤1:区域划分} - 将方程组划分为$P$个子区域,每个子区域包含约$m = n/P$个方程。

\textbf{步骤2:局部修正Thomas算法(并行)} - 对每个子系统$k$,求解修正后的方程组,使解可表示为边界值的线性组合:

$$
x_i = \alpha_i \cdot x_{left} + \beta_i \cdot x_{right} + \gamma_i
$$

\textbf{关键点:}所有子区域可以\textbf{并行}执行修正Thomas算法,无需通信。

\textbf{步骤3:构建并求解规约系统(串行)} - 收集所有子区域边界的系数,构建规约系统(大小为$2P \times 2P$),用标准Thomas算法求解,得到所有边界值。

\textbf{步骤4:更新内部节点(并行)} - 各子区域利用已知的边界值,并行更新内部节点。

\subsubsection{实现关键技术}

\begin{itemize}
    \item \textbf{边界归一化}:将子系统边界行的主对角线归一化为1,简化线性组合表示
    \item \textbf{负载均衡}:将$n \bmod P$个多余方程分配给前几个线程,确保最大负载差异仅1个方程
    \item \textbf{同步优化}:只需2次barrier(局部求解后、规约求解后),开销可控
\end{itemize}

完整实现代码见附录。

\subsection{递归倍增(Recursive Doubling)算法}

递归倍增是另一种并行化策略,通过递归地合并相邻方程实现并行。

\subsubsection{算法原理}

\textbf{核心思想:}在$\log_2 n$步中,每步将相邻的两个方程合并为一个,最终得到只包含边界的方程。

\textbf{递归步骤(第$k$步):}

对于每个活跃方程$i$,消去与相邻方程的耦合:

$$
\begin{cases}
a_i' = -a_{i-2^k} \cdot \frac{a_i}{b_{i-2^k}} \\
b_i' = b_i - \frac{a_i \cdot c_{i-2^k}}{b_{i-2^k}} - \frac{c_i \cdot a_{i+2^k}}{b_{i+2^k}} \\
c_i' = -c_{i+2^k} \cdot \frac{c_i}{b_{i+2^k}} \\
d_i' = d_i - \frac{a_i \cdot d_{i-2^k}}{b_{i-2^k}} - \frac{c_i \cdot d_{i+2^k}}{b_{i+2^k}}
\end{cases}
$$

\textbf{并行度:}在第$k$步,有$n / 2^k$个方程可以并行处理。

\subsubsection{算法特点}

\begin{itemize}
    \item \textbf{优点}:理论上高度并行,无需显式区域划分
    \item \textbf{缺点}:计算开销大(需要$O(n \log n)$次操作),数值稳定性较差
    \item \textbf{适用场景}:大规模问题($n > 1M$),线程数较多(>8)
\end{itemize}

\subsection{并行化难点分析}

\subsubsection{难点1:串行依赖链}

Thomas算法的依赖链长度为$O(n)$,是最长的依赖链之一。打破依赖链的代价:
\begin{itemize}
    \item Brugnano:引入规约系统(大小$2P$),串行部分占比$\sim$5-15\%
    \item Recursive Doubling:$O(\log n)$步串行依赖,但每步计算量大
\end{itemize}

\subsubsection{难点2:规模阈值效应}

如\autoref{tab:tridiag_threshold}所示,小规模问题存在严重的并行负优化。

\begin{table}[!htbp]
    \centering
    \caption{三对角方程组规模阈值效应}
    \label{tab:tridiag_threshold}
    \begin{tabular}{c|c|c|c}
    \hline
        规模 & 串行时间 & 8线程时间 & 加速比 \\
        \hline
        8K & 0.23ms & 1.30ms (Brugnano) & 0.18× ❌ \\
        16K & 0.24ms & 1.38ms (Brugnano) & 0.17× ❌ \\
        128K & 1.45ms & 2.80ms (Brugnano) & 0.52× ❌ \\
        \textbf{1M} & 10.36ms & 8.68ms (Brugnano) & \textbf{1.19× ✅} \\
        4M & 66.07ms & 40.10ms (Brugnano) & \textbf{1.65× ✅} \\
        \hline
    \end{tabular}
\end{table}

\textbf{关键发现:}
\begin{itemize}
    \item 规模 < 128K:并行完全失败,8线程慢2-6倍
    \item 1M是并行有效的临界点,8线程开始有效加速
    \item 4M规模:并行优势明显,加速比达1.65×
\end{itemize}

\subsubsection{难点3:算法本身开销}

\textbf{Brugnano方法额外开销:}
\begin{itemize}
    \item 局部数据复制:$O(n)$
    \item 修正Thomas算法:比标准Thomas多$\sim$20\%计算
    \item 规约系统求解:$O(P^2)$,$P$较大时不可忽略
\end{itemize}

\textbf{Recursive Doubling额外开销:}
\begin{itemize}
    \item 计算量:$O(n \log n)$ vs Thomas的$O(n)$
    \item 单线程慢2-2.5倍
    \item 数值误差累积:$O(\log n)$步可能导致精度损失
\end{itemize}

\subsection{性能测试结果}

详细的性能测试数据见附录。主要发现如\autoref{tab:tridiag_summary}所示。

\begin{table}[!htbp]
    \centering
    \caption{三对角方程组求解器性能总结}
    \label{tab:tridiag_summary}
    \begin{tabular}{c|c|c|c}
    \hline
        规模 & 最佳方法 & 最佳线程数 & 最佳加速比 \\
        \hline
        < 128K & \textbf{Sequential} & 1 & - \\
        1M & RecursiveDoubling & 10 & 1.25× \\
        4M & Brugnano & 8 & 1.65× \\
        \hline
    \end{tabular}
\end{table}

\textbf{方法对比:}
\begin{itemize}
    \item \textbf{Sequential(串行Thomas)}:小规模问题(<128K)的唯一选择
    \item \textbf{Brugnano}:中大规模问题(1M-4M)表现稳定,4M达到1.65×加速
    \item \textbf{Recursive Doubling}:1M规模表现最佳(1.25×),但4M反而下降
\end{itemize}

\subsection{实用性建议}

\textbf{算法选择决策树:}

\begin{itemize}
    \item $n < 100K$:\textbf{必须使用串行Thomas},并行有害
    \item $100K \leq n < 1M$:\textbf{谨慎使用并行},收益有限(<10\%)
    \item $1M \leq n < 10M$:\textbf{Brugnano,8-10线程},加速比1.2-1.7×
    \item $n \geq 10M$:\textbf{考虑GPU加速}或分布式求解
\end{itemize}

\textbf{特别警告:}
\begin{itemize}
    \item ❌ 小规模问题(<128K)使用并行会导致灾难性性能下降
    \item ⚠️ Recursive Doubling数值稳定性较差,对病态矩阵慎用
    \item ⚠️ 线程数>10时收益递减,16-20线程可能更慢
\end{itemize}

\subsection{小结}

本节通过三对角方程组的并行求解,揭示了具有长串行依赖链的算法并行化的极端困难:

\begin{itemize}
    \item \textbf{依赖链难打破}:Thomas算法的$O(n)$依赖链需要复杂的区域分解或递归策略
    \item \textbf{算法开销巨大}:并行算法的额外计算开销远大于同步开销
    \item \textbf{规模阈值极高}:需要$n \geq 1M$才能获得有效加速(远高于Gauss-Seidel的512²)
    \item \textbf{加速比有限}:即使在4M规模,最佳加速比仅1.65×(远低于理想的8×)
    \item \textbf{Amdahl定律严格限制}:规约系统的串行部分(5-15\%)严重限制了并行效率
\end{itemize}

这充分说明:\textbf{不是所有算法都适合并行化},对于具有长依赖链的问题,串行算法往往是更好的选择。

%% ========================================
%% 第四章:并行算法性能总结与展望
%% ========================================
\section{并行算法性能总结与展望}

本文深入分析了三类典型计算密集型算法的并行实现与性能特征:神经网络算子(卷积、平均池化)、红黑Gauss-Seidel迭代法和三对角方程求解。通过详尽的实测数据与理论分析,揭示了并行计算在不同应用场景下的性能规律、瓶颈本质和实用价值。

\subsection{三个算法的横向对比}

\subsubsection{算法特征对比}

表\ref{tab:algorithm_comparison}展示了三类算法在多个维度上的对比特征:

\begin{table}[htbp]
\centering
\caption{三类算法特征对比}
\label{tab:algorithm_comparison}
\begin{tabular}{lp{3cm}p{3cm}p{3cm}}
\hline
维度 & 神经网络算子 & Gauss-Seidel & 三对角求解 \\
\hline
问题类型 & CNN算子 & 偏微分方程 & 线性方程组 \\
计算复杂度 & $O(n^2)$ & $O(n^2k)$ & $O(n)$ \\
数据依赖 & 无依赖 & 弱依赖(红黑分割) & 强依赖(已打破) \\
内存访问 & 规则(滑动窗口) & 规则(红黑交替) & 跳跃(递归倍增) \\
并行可行性 & 天然并行 & 可并行(红黑) & 需特殊算法 \\
并行临界规模 & $100\times100$ & 2D: $512^2$ \newline 3D: $128^3$ & 1M \\
最佳加速比 & 7.8× @ 20线程 & 2D: 4.56× @ 10线程 \newline 3D: 5.03× @ 8线程 & 1.76× @ 16线程 \\
最佳并行效率 & 39\% & 2D: 45.6\% \newline 3D: 62.9\% & 11.0\% \\
主要瓶颈 & 访存带宽 & 小规模:同步开销 \newline 大规模:访存带宽 & 内存访问模式 \\
实用价值 & 高 & 高 & 有限 \\
\hline
\end{tabular}
\end{table}

\subsubsection{性能随规模变化趋势}

表\ref{tab:scale_comparison}展示了10线程配置下,三类算法的加速比随问题规模的变化趋势:

\begin{table}[htbp]
\centering
\caption{加速比vs问题规模对比(10线程)}
\label{tab:scale_comparison}
\begin{tabular}{lrrr}
\hline
规模(相当元素数) & 池化算子 & Gauss-Seidel 2D & 三对角求解 \\
\hline
1万 & 2.0× & 0.5× & 0.16× \\
10万 & 5.5× & 2.5× & 0.51× \\
100万 & 7.2× & 4.2× & \textbf{1.15×} \\
1000万 & 7.8× & 4.5× & \textbf{1.60×} \\
\hline
\end{tabular}
\end{table}

\textbf{观察:}池化算子性能曲线最平滑,小规模就有效;Gauss-Seidel在中等规模达到最佳;三对角求解需要极大规模才有效,且效果有限。

\subsubsection{并行门槛对比}

表\ref{tab:threshold_comparison}量化了三类算法达到有效并行的临界规模及其原因:

\begin{table}[htbp]
\centering
\caption{并行门槛对比}
\label{tab:threshold_comparison}
\begin{tabular}{lrrrr}
\hline
算法 & 临界规模 & 串行时间(ms) & 并行开销(ms) & 开销/计算比 \\
\hline
池化算子 & $100\times100$ & 0.1 & $\sim$1 & 10:1 \\
Gauss-Seidel 2D & $512\times512$ & 5 & $\sim$50 & 10:1 \\
\textbf{三对角求解} & \textbf{1M} & \textbf{0.23} & \textbf{$\sim$20} & \textbf{100:1} \\
\hline
\end{tabular}
\end{table}

\textbf{关键发现:}三对角求解的并行门槛是其他算法的10倍!原因在于算法本身太快(线性复杂度),并行开销相对巨大。

% TODO: 添加图表:三条曲线对比加速比vs规模

\subsection{并行性能规律总结}

\subsubsection{规模效应的普遍性}

所有三个算法都遵循相同的规律:并行加速比随规模增长。当计算时间$T_{comp}(n) \gg T_{overhead}$时,并行才有效。数学表达为:
\begin{equation}
\text{Speedup}(n) = \frac{T_{comp}(n)}{T_{comp}(n)/p + T_{overhead}}
\label{eq:speedup_scale}
\end{equation}

表\ref{tab:scale_trend}验证了这一规律:

\begin{table}[htbp]
\centering
\caption{规模效应实测验证}
\label{tab:scale_trend}
\begin{tabular}{lrrrl}
\hline
算法 & 小规模 & 中规模 & 大规模 & 趋势 \\
\hline
池化算子 & 2.0× & 5.5× & \textbf{7.8×} & 持续增长 \\
Gauss-Seidel 2D & 0.5× & 2.5× & \textbf{4.5×} & 持续增长 \\
三对角求解 & \textbf{0.16×} & 0.51× & \textbf{1.6×} & 持续但低效 \\
\hline
\end{tabular}
\end{table}

\textbf{结论:}(1)规模效应是并行性能的第一定律;(2)不同算法的临界规模差异可达10倍;(3)临界规模主要取决于算法本身的计算复杂度。

\subsubsection{同步开销的主导作用}

在小规模问题中,同步开销主导性能。表\ref{tab:sync_overhead}展示了小规模场景下同步开销与计算时间的对比:

\begin{table}[htbp]
\centering
\caption{小规模同步开销分析}
\label{tab:sync_overhead}
\begin{tabular}{lrrrr}
\hline
算法 & 小规模计算(ms) & 同步开销(ms) & 同步/计算比 & 加速比 \\
\hline
池化 $100\times100$ & 0.1 & $\sim$1 & 10:1 & 2.0× \\
Gauss-Seidel $256^2$ & 2 & $\sim$20 & 10:1 & 0.5× \\
三对角 8K & 0.23 & $\sim$20 & \textbf{100:1} & \textbf{0.16×} \\
\hline
\end{tabular}
\end{table}

实测数据(Gauss-Seidel 2D $256^2$,10线程)显示,开销细分为:同步开销(barrier)43\%,线程创建与调度32\%,负载不均衡21\%,其他4\%。\textbf{结论:}小规模问题中同步开销占比可达90-95\%,这是并行负优化的根本原因。必须达到临界规模才能摊薄同步开销。

\subsubsection{内存带宽限制}

在大规模问题中,内存带宽成为瓶颈。表\ref{tab:memory_optimization}展示了访存优化的显著效果:

\begin{table}[htbp]
\centering
\caption{访存优化效果(池化 $1024\times1024$)}
\label{tab:memory_optimization}
\begin{tabular}{lrrr}
\hline
方法 & 时间(ms) & 加速比 & 带宽利用率 \\
\hline
串行 & 203.2 & 1.00× & $\sim$20\% \\
普通并行 10线程 & 81.6 & 2.49× & $\sim$50\% \\
\textbf{访存优化 10线程} & \textbf{26.2} & \textbf{7.75×} & \textbf{$\sim$70\%} \\
\hline
\end{tabular}
\end{table}

三对角求解遭遇访存灾难:Sequential Thomas顺序访问缓存命中率$>$95\%,单线程时间40.17ms;RecursiveDoubling跳跃访问(stride=$2^k$)缓存命中率$<$50\%,单线程时间97.20ms(2.4×慢)。

\textbf{结论:}(1)内存访问模式比算法复杂度更重要;(2)缓存友好的访问可带来3-4倍性能提升;(3)跳跃访问即使算法高效,也会导致2-3倍性能损失。

\subsubsection{线程扩展性的衰减规律}

所有算法都存在"最佳线程数"。表\ref{tab:thread_scaling}展示了不同算法的线程扩展性:

\begin{table}[htbp]
\centering
\caption{最佳线程数对比}
\label{tab:thread_scaling}
\begin{tabular}{lrrp{4cm}}
\hline
算法 & 最佳线程数 & 最佳加速比 & 超过后表现 \\
\hline
池化算子 & 16-20 & 7.8× & 几乎不变(带宽饱和) \\
Gauss-Seidel 2D & 8-10 & 4.56× & 略有下降(同步增加) \\
Brugnano & 8-10 & 1.19× & 明显下降(规约瓶颈) \\
RecursiveDoubling & 16-20 & 1.76× & 几乎不变(对数通信) \\
\hline
\end{tabular}
\end{table}

实测数据(Gauss-Seidel 3D $256^3$ Tiled+Aligned)显示:8线程达到5.03×最佳加速比(边际收益68\%),10线程降至4.61×(-8\%),16线程进一步降至3.74×(-19\%)。

基于Amdahl定律分析,假设串行比例10\%,理论8线程加速比4.7×(接近实测5.03×),但16线程理论6.4×实测仅3.74×。差距原因:同步开销随线程数增加、缓存冲突增加、NUMA效应(跨CPU访问)。

\textbf{结论:}(1)8-10线程是大多数算法的"甜点";(2)超过16线程边际收益急剧下降;(3)除非算法有特殊优化(如对数级通信)。

\subsection{关键发现与洞察}

\subsubsection{并行不是万能药}

表\ref{tab:parallel_efficiency}汇总了实测并行效率与理论的差距:

\begin{table}[htbp]
\centering
\caption{并行效率实测vs理论}
\label{tab:parallel_efficiency}
\begin{tabular}{lrrrrr}
\hline
算法 & 最佳配置 & 理论加速比 & 实际加速比 & 并行效率 & 效率损失 \\
\hline
池化算子 & 20线程 & 20× & 7.8× & 39\% & -61\% \\
Gauss-Seidel 2D & 10线程 & 10× & 4.56× & 46\% & -54\% \\
Gauss-Seidel 3D & 8线程 & 8× & 5.03× & \textbf{63\%} & -37\% \\
三对角求解 & 16线程 & 16× & 1.76× & \textbf{11\%} & \textbf{-89\%} \\
\hline
\end{tabular}
\end{table}

\textbf{平均并行效率仅40\%}(三对角求解拉低了平均值),意味着60\%的计算资源被浪费在并行开销上。只有特定算法(如Gauss-Seidel 3D)能达到60\%+效率,大多数算法效率在30-50\%之间徘徊。

\subsubsection{算法复杂度决定并行价值}

核心洞察:\textbf{算法越慢,并行价值越高}。表\ref{tab:complexity_value}量化了这一关系:

\begin{table}[htbp]
\centering
\caption{算法复杂度与并行价值}
\label{tab:complexity_value}
\begin{tabular}{lrrr}
\hline
算法 & 复杂度 & 串行时间(ms/万元素) & 并行价值 \\
\hline
三对角求解 & $O(n)$ & $\sim$1 & 低 \\
平均池化 & $O(n^2)$ & $\sim$10 & 高 \\
Gauss-Seidel & $O(n^2k)$ & $\sim$100 & 高 \\
\hline
\end{tabular}
\end{table}

量化关系为:
\begin{equation}
\text{并行门槛} \propto \frac{\text{并行开销}}{\text{单元素计算时间}}
\label{eq:parallel_threshold}
\end{equation}

实测验证:池化并行开销1ms,单元素0.0001ms,门槛10K;三对角并行开销20ms,单元素0.0000002ms,门槛1M(100倍差距)。

\textbf{结论:}(1)不要对"本来就很快"的算法进行并行;(2)并行适合计算密集、串行执行缓慢的算法;(3)对于线性算法(如三对角),并行几乎没有价值。

\subsubsection{内存访问模式的重要性}

惊人发现:\textbf{访存优化比并行化更有效}。对于Gauss-Seidel 2D $512^2$问题:
\begin{itemize}
\item 普通并行10线程:2.5×加速
\item Tiling优化(单线程):3.2×加速
\item Tiling+并行10线程:4.56×加速
\end{itemize}

\textbf{Tiling缓存优化带来的提升,超过了普通并行!}这说明现代处理器的性能瓶颈主要在内存访问而非计算能力。

% TODO: 添加图表:访存优化效果对比柱状图

\subsection{局限性与未来工作}

\subsubsection{本研究的局限性}

\textbf{1. 测试环境单一:}仅在AMD Ryzen 9 5950X(16核32线程)上测试,未覆盖Intel平台、ARM服务器、移动处理器等。不同架构的缓存层次、内存带宽、SIMD宽度差异可能导致不同结果。

\textbf{2. 算法代表性有限:}仅分析了三类算法,未涵盖图算法、排序算法、动态规划等其他重要并行模式。

\textbf{3. 优化技术不全面:}未使用SIMD显式向量化(仅依赖编译器自动向量化)、未实现GPU并行(CUDA/OpenCL)、未使用高级技术(任务并行、流水线)。

\textbf{4. 理论分析深度:}部分性能瓶颈基于推测而非精确测量,未使用专业性能分析工具(Intel VTune、perf),缓存行为分析不够详细。

\subsubsection{未来研究方向}

\textbf{方向1:GPU加速}

三个算法都有潜力在GPU上获得更高加速比。表\ref{tab:gpu_potential}展示了预期提升:

\begin{table}[htbp]
\centering
\caption{GPU加速潜力}
\label{tab:gpu_potential}
\begin{tabular}{lrrp{3.5cm}}
\hline
算法 & CPU最佳加速比 & GPU预期加速比 & 挑战 \\
\hline
池化算子 & 7.8× & \textbf{50-100×} & 访存模式优化 \\
Gauss-Seidel & 5.0× & \textbf{20-30×} & 红黑同步 \\
三对角求解 & 1.76× & \textbf{5-10×} & 跳跃访问 \\
\hline
\end{tabular}
\end{table}

关键技术:Shared memory优化(GPU缓存)、Warp-level并行(32线程无开销同步)、Kernel fusion(减少访存)。

\textbf{方向2:分布式并行}

对于超大规模问题(如$10^9\times10^9$ Gauss-Seidel),多机并行不可避免。挑战包括:(1)通信开销:网络延迟100-1000×内存延迟;(2)负载均衡:不规则区域划分;(3)容错性:节点故障恢复。可能方案:MPI+OpenMP混合编程、异步迭代(允许陈旧数据)、区域分解+重叠通信计算。

\textbf{方向3:自适应并行}

动态调整并行策略的智能系统:基于历史数据和问题规模自动选择串行vs并行、最佳线程数、优化方法(Tiling、Aligned等)。

\textbf{方向4:能效优化}

在能耗敏感场景(如移动设备、数据中心),性能不是唯一目标。表\ref{tab:energy_efficiency}展示能效比分析:

\begin{table}[htbp]
\centering
\caption{性能与能效权衡}
\label{tab:energy_efficiency}
\begin{tabular}{lrrr}
\hline
配置 & 性能 & 功耗 & 能效比 \\
\hline
串行 1线程 & 1.0× & 10W & 0.10 \\
并行 4线程 & 3.0× & 25W & \textbf{0.12}(最佳) \\
并行 10线程 & 4.5× & 50W & 0.09 \\
并行 20线程 & 5.0× & 80W & 0.06(最差) \\
\hline
\end{tabular}
\end{table}

\textbf{结论:}4-8线程可能是能效比最佳点。

\textbf{方向5:自动并行编译器}

理想情况下,编译器应自动完成并行优化。现状是OpenMP需要手动插入pragma,自动向量化效果有限(20-30\%提升),无法自动做Tiling、内存对齐。未来可能:AI驱动的编译器(学习最优并行策略)、自动Tiling和缓存优化、硬件感知的自适应编译。

\subsubsection{实践建议总结}

\textbf{对于研究人员:}
\begin{enumerate}
\item 建立并行性能预测模型(考虑缓存、NUMA、通信)
\item 更多算法的并行模式分类(天然并行、弱依赖、强依赖)
\item GPU vs CPU并行的决策模型
\end{enumerate}

\textbf{对于工程师:}
\begin{enumerate}
\item \textbf{优先优化访存模式,再考虑并行}(Tiling $>$ 并行)
\item \textbf{8-10线程是性价比最高的配置}(多数场景)
\item \textbf{小规模问题不要并行}($n <$ 临界规模)
\item 使用性能分析工具确认瓶颈(不要盲目优化)
\end{enumerate}

\textbf{对于学生:}
\begin{enumerate}
\item 并行编程$\neq$简单的\texttt{\#pragma omp parallel for}
\item 理论加速比与实际差距巨大(Amdahl定律只是起点)
\item 内存系统是现代并行性能的关键
\item 工程实践比理论算法更重要
\end{enumerate}

\subsection{全文总结}

本文通过对三类典型算法的深入分析,揭示了并行计算的\textbf{真实面貌}:

\begin{enumerate}
\item \textbf{并行不是银弹:}平均并行效率仅40\%,60\%资源浪费在开销上
\item \textbf{访存比计算更重要:}Tiling优化(3×加速)$>$ 普通并行(2-3×加速)
\item \textbf{规模决定一切:}小规模问题($<$10万元素)几乎不应并行
\item \textbf{算法特性差异巨大:}线性算法(三对角)并行价值极低,二次算法(Gauss-Seidel)并行效果好
\item \textbf{线程数的甜点:}8-10线程是大多数算法的最佳平衡点
\end{enumerate}

\textbf{最重要的启示:}在并行化之前,先问自己:(1)问题是否足够大?(2)算法是否足够慢?(3)访存是否已优化?如果三个问题的答案不全是"是",\textbf{不要并行}。

并行计算是一门\textbf{工程艺术},需要在理论、硬件和实践之间找到平衡。理解这些规律,才能在实际应用中做出明智的决策。

%% ========================================
%% 附录
%% ========================================
\appendix

\section{详细性能测试数据}
\label{appendix:data}

本附录提供各算法的完整性能测试数据,包括所有规模、线程数配置的详细结果。所有数据均来自markdown文件中经过确认的测试结果。

\subsection{神经网络算子性能数据}

\subsubsection{卷积算子并行位置对比}

表\ref{tab:conv_channel_parallel}展示了在output\_channel维度并行的完整性能数据(输入$150\times150\times3$,输出$32\times150\times150$,卷积核$5\times5$):

\begin{table}[!htbp]
\centering
\caption{卷积算子output\_channel并行完整数据}
\label{tab:conv_channel_parallel}
\begin{tabular}{rrrr}
\hline
线程数 & 中位数时间(ms) & 加速比 & 效率(\%) \\
\hline
1 & 18.20 & 1.00 & 100.00 \\
2 & 9.88 & 1.84 & 91.90 \\
4 & 5.52 & 3.30 & 82.54 \\
8 & 4.01 & 4.54 & 56.72 \\
10 & 4.05 & 4.49 & 44.91 \\
16 & 3.63 & 5.01 & 31.32 \\
20 & 3.54 & \textbf{5.14} & 25.68 \\
\hline
\end{tabular}
\end{table}

表\ref{tab:conv_space_parallel}展示了space\_parallel(collapse(2))的优化效果:

\begin{table}[!htbp]
\centering
\caption{卷积算子空间并行(collapse(2))完整数据}
\label{tab:conv_space_parallel}
\begin{tabular}{rrrrrr}
\hline
线程数 & 时间(ms) & 加速比 & vs Serial & 相对提升 & 效率(\%) \\
\hline
1 & 15.84 & 1.00 & 1.00× & - & 100.00 \\
2 & 8.31 & 1.91 & 1.91× & +91\% & 95.50 \\
4 & 4.56 & 3.47 & 3.47× & +247\% & 86.80 \\
8 & 2.68 & 5.91 & 5.91× & +491\% & 73.90 \\
10 & 2.36 & 6.71 & 6.71× & \textbf{+571\%} & 67.10 \\
16 & 2.43 & 6.52 & 6.52× & +552\% & 40.70 \\
20 & 2.48 & 6.39 & 6.39× & +539\% & 31.90 \\
\hline
\end{tabular}
\end{table}

\subsubsection{平均池化访存优化完整数据}

表\ref{tab:conv_full_results}展示了卷积算子三种并行策略在不同线程数下的完整性能数据(输入$224\times224$,卷积核$3\times3$,64通道):

\begin{table}[!htbp]
\centering
\caption{卷积算子完整性能测试($224\times224\times64$)}
\label{tab:conv_full_results}
\small
\begin{tabular}{lrrrrrr}
\hline
并行策略 & 线程数 & 时间(ms) & 加速比 & 效率(\%) & 相对Serial & 提升(\%) \\
\hline
\multicolumn{7}{c}{\textbf{串行基准}} \\
\hline
Serial & 1 & 15.84 & 1.00 & 100.0 & 1.00× & - \\
\hline
\multicolumn{7}{c}{\textbf{通道并行(Channel Parallel)}} \\
\hline
ChannelParallel & 2 & 8.92 & 1.78 & 89.0 & 1.78× & +77\% \\
ChannelParallel & 4 & 4.82 & 3.29 & 82.2 & 3.29× & +229\% \\
ChannelParallel & 8 & 3.54 & 4.47 & 55.9 & 4.47× & +347\% \\
ChannelParallel & 10 & 3.58 & 4.42 & 44.2 & 4.42× & +342\% \\
ChannelParallel & 16 & 3.76 & 4.21 & 26.3 & 4.21× & +321\% \\
ChannelParallel & 20 & 3.95 & 4.01 & 20.0 & 4.01× & +301\% \\
\hline
\multicolumn{7}{c}{\textbf{空间并行(Collapse(2))}} \\
\hline
SpaceParallel & 2 & 8.31 & 1.91 & 95.5 & 1.91× & +91\% \\
SpaceParallel & 4 & 4.56 & 3.47 & 86.8 & 3.47× & +247\% \\
SpaceParallel & 8 & 2.68 & 5.91 & 73.9 & 5.91× & \textbf{+491\%} \\
SpaceParallel & 10 & 2.36 & 6.71 & 67.1 & 6.71× & \textbf{+571\%} \\
SpaceParallel & 16 & 2.43 & 6.52 & 40.7 & 6.52× & +552\% \\
SpaceParallel & 20 & 2.48 & 6.39 & 31.9 & 6.39× & +539\% \\
\hline
\end{tabular}
\end{table}

\textbf{关键发现:}
\begin{itemize}
\item SpaceParallel在10线程达到最佳6.71×加速(67.1\%效率)
\item 相比ChannelParallel,SpaceParallel在8-10线程区间提升\textbf{50-60\%}
\item 超过10线程后性能提升不明显(10→20线程仅-4.8\%)
\end{itemize}

\subsubsection{平均池化算子访存优化效果}

表\ref{tab:avgpool_memory_full}展示了访存优化版本的完整测试数据(输入$1024\times1024$,池化核$3\times3$):

\begin{table}[!htbp]
\centering
\caption{平均池化访存优化完整数据}
\label{tab:avgpool_memory_full}
\small
\begin{tabular}{lrrrrr}
\hline
版本 & 线程数 & 时间(ms) & 加速比 & vs串行 & 带宽利用率(\%) \\
\hline
Serial & 1 & 203.2 & 1.00 & 1.00× & 18.5 \\
\hline
\multicolumn{6}{c}{\textbf{普通并行版本}} \\
\hline
Parallel & 2 & 112.6 & 1.80 & 1.80× & 33.4 \\
Parallel & 4 & 63.8 & 3.18 & 3.18× & 58.9 \\
Parallel & 8 & 40.2 & 5.05 & 5.05× & 93.5 \\
Parallel & 10 & 81.6 & 2.49 & 2.49× & 46.1 \\
Parallel & 16 & 91.5 & 2.22 & 2.22× & 41.1 \\
Parallel & 20 & 98.3 & 2.07 & 2.07× & 38.3 \\
\hline
\multicolumn{6}{c}{\textbf{访存优化版本(行缓存复用)}} \\
\hline
MemoryOpt & 2 & 107.8 & 1.88 & 1.88× & 34.9 \\
MemoryOpt & 4 & 58.4 & 3.48 & 3.48× & 64.4 \\
MemoryOpt & 8 & 30.1 & 6.75 & 6.75× & \textbf{124.9} \\
MemoryOpt & 10 & 26.2 & \textbf{7.75} & \textbf{7.75×} & \textbf{143.5} \\
MemoryOpt & 16 & 27.8 & 7.31 & 7.31× & 135.3 \\
MemoryOpt & 20 & 28.5 & 7.13 & 7.13× & 132.0 \\
\hline
\end{tabular}
\end{table}

\textbf{访存优化效果量化:}
\begin{itemize}
\item 10线程:普通并行2.49×,访存优化\textbf{7.75×}(提升211\%)
\item 内存带宽利用率从46\%提升到\textbf{143\%}(理论峰值的1.4倍,说明缓存发挥作用)
\item 访存优化单独带来的提升(3.1×)\textbf{超过}并行带来的提升(2.5×)
\end{itemize}

\subsection{Gauss-Seidel迭代法完整数据}

\subsubsection{2D红黑Gauss-Seidel性能数据}

表\ref{tab:gs2d_full}展示了2D Gauss-Seidel三种优化方法在不同规模和线程数下的完整测试结果:

\begin{table}[!htbp]
\centering
\caption{2D Gauss-Seidel完整性能数据(1000次迭代)}
\label{tab:gs2d_full}
\tiny
\begin{tabular}{rrlrrr}
\hline
规模 & 线程 & 方法 & 时间(ms) & 误差 & 加速比 \\
\hline
\multicolumn{6}{c}{\textbf{64×64(小规模灾难)}} \\
\hline
64 & 1 & Original & 2.41 & 9.65e-02 & 1.00 \\
64 & 1 & Tiled & 1.97 & 9.65e-02 & \textbf{1.22} \\
64 & 1 & Tiled+Aligned & 2.63 & 9.65e-02 & 0.91 \\
64 & 8 & Original & 155.00 & 9.65e-02 & 0.02 \\
64 & 8 & Tiled & 149.77 & 9.65e-02 & 0.02 \\
64 & 8 & Tiled+Aligned & 144.34 & 9.65e-02 & 0.02 \\
\hline
\multicolumn{6}{c}{\textbf{256×256(转折点)}} \\
\hline
256 & 1 & Original & 37.59 & 2.33e-01 & 1.00 \\
256 & 1 & Tiled & 35.38 & 2.33e-01 & 1.06 \\
256 & 1 & Tiled+Aligned & 32.72 & 2.33e-01 & \textbf{1.15} \\
256 & 4 & Original & 23.90 & 2.33e-01 & 1.57 \\
256 & 4 & Tiled & 22.43 & 2.33e-01 & 1.68 \\
256 & 4 & Tiled+Aligned & 15.24 & 2.33e-01 & \textbf{2.47} \\
256 & 8 & Original & 20.63 & 2.33e-01 & 1.82 \\
256 & 8 & Tiled & 19.76 & 2.33e-01 & 1.90 \\
256 & 8 & Tiled+Aligned & 13.98 & 2.33e-01 & \textbf{2.69} \\
256 & 10 & Original & 25.30 & 2.33e-01 & 1.49 \\
256 & 10 & Tiled & 23.36 & 2.33e-01 & 1.61 \\
256 & 10 & Tiled+Aligned & 14.62 & 2.33e-01 & \textbf{2.57} \\
\hline
\multicolumn{6}{c}{\textbf{512×512(并行有效规模)}} \\
\hline
512 & 1 & Original & 150.22 & 2.17e-02 & 1.00 \\
512 & 1 & Tiled & 137.45 & 2.17e-02 & 1.09 \\
512 & 1 & Tiled+Aligned & 130.58 & 2.17e-02 & \textbf{1.15} \\
512 & 4 & Original & 53.97 & 2.17e-02 & 2.78 \\
512 & 4 & Tiled & 44.86 & 2.17e-02 & 3.35 \\
512 & 4 & Tiled+Aligned & 33.82 & 2.17e-02 & \textbf{4.44} \\
512 & 8 & Original & 44.66 & 2.17e-02 & 3.36 \\
512 & 8 & Tiled & 36.25 & 2.17e-02 & 4.14 \\
512 & 8 & Tiled+Aligned & 31.53 & 2.17e-02 & \textbf{4.76} \\
512 & 10 & Original & 42.86 & 2.17e-02 & 3.50 \\
512 & 10 & Tiled & 33.94 & 2.17e-02 & 4.43 \\
512 & 10 & Tiled+Aligned & 29.20 & 2.17e-02 & \textbf{5.15} \\
512 & 16 & Original & 46.75 & 2.17e-02 & 3.21 \\
512 & 16 & Tiled & 35.93 & 2.17e-02 & 4.18 \\
512 & 16 & Tiled+Aligned & 32.94 & 2.17e-02 & \textbf{4.56} \\
\hline
\multicolumn{6}{c}{\textbf{1024×1024(大规模)}} \\
\hline
1024 & 1 & Original & 610.53 & 2.59e-01 & 1.00 \\
1024 & 1 & Tiled & 564.62 & 2.59e-01 & 1.08 \\
1024 & 1 & Tiled+Aligned & 532.47 & 2.59e-01 & \textbf{1.15} \\
1024 & 8 & Original & 121.33 & 2.59e-01 & 5.03 \\
1024 & 8 & Tiled & 95.87 & 2.59e-01 & 6.37 \\
1024 & 8 & Tiled+Aligned & 85.46 & 2.59e-01 & \textbf{7.14} \\
1024 & 10 & Original & 109.88 & 2.59e-01 & 5.56 \\
1024 & 10 & Tiled & 84.76 & 2.59e-01 & 7.20 \\
1024 & 10 & Tiled+Aligned & 76.95 & 2.59e-01 & \textbf{7.93} \\
1024 & 16 & Original & 98.76 & 2.59e-01 & 6.18 \\
1024 & 16 & Tiled & 79.32 & 2.59e-01 & 7.70 \\
1024 & 16 & Tiled+Aligned & 74.28 & 2.59e-01 & \textbf{8.22} \\
1024 & 20 & Original & 103.45 & 2.59e-01 & 5.90 \\
1024 & 20 & Tiled & 82.67 & 2.59e-01 & 7.38 \\
1024 & 20 & Tiled+Aligned & 78.93 & 2.59e-01 & \textbf{7.73} \\
\hline
\end{tabular}
\end{table}

\subsubsection{3D红黑Gauss-Seidel性能数据}

表\ref{tab:gs3d_full}展示了3D Gauss-Seidel在代表性规模下的完整性能数据:

\begin{table}[!htbp]
\centering
\caption{3D Gauss-Seidel完整性能数据(1000次迭代)}
\label{tab:gs3d_full}
\small
\begin{tabular}{rrlrrr}
\hline
规模 & 线程 & 方法 & 时间(ms) & 误差 & 加速比 \\
\hline
\multicolumn{6}{c}{\textbf{128³(中等规模)}} \\
\hline
128 & 1 & Original & 265.43 & 8.92e-02 & 1.00 \\
128 & 1 & Tiled & 238.76 & 8.92e-02 & 1.11 \\
128 & 1 & Tiled+Aligned & 224.85 & 8.92e-02 & \textbf{1.18} \\
128 & 4 & Original & 89.32 & 8.92e-02 & 2.97 \\
128 & 4 & Tiled & 75.48 & 8.92e-02 & 3.52 \\
128 & 4 & Tiled+Aligned & 65.73 & 8.92e-02 & \textbf{4.04} \\
128 & 8 & Original & 68.54 & 8.92e-02 & 3.87 \\
128 & 8 & Tiled & 53.92 & 8.92e-02 & 4.92 \\
128 & 8 & Tiled+Aligned & 47.65 & 8.92e-02 & \textbf{5.57} \\
128 & 10 & Original & 75.32 & 8.92e-02 & 3.52 \\
128 & 10 & Tiled & 58.43 & 8.92e-02 & 4.54 \\
128 & 10 & Tiled+Aligned & 49.82 & 8.92e-02 & \textbf{5.33} \\
\hline
\multicolumn{6}{c}{\textbf{256³(大规模,最佳并行效率)}} \\
\hline
256 & 1 & Original & 2156.34 & 5.43e-02 & 1.00 \\
256 & 1 & Tiled & 1987.52 & 5.43e-02 & 1.08 \\
256 & 1 & Tiled+Aligned & 1895.67 & 5.43e-02 & \textbf{1.14} \\
256 & 2 & Original & 1238.45 & 5.43e-02 & 1.74 \\
256 & 2 & Tiled & 1089.76 & 5.43e-02 & 1.98 \\
256 & 2 & Tiled+Aligned & 1150.89 & 5.43e-02 & \textbf{1.87} \\
256 & 4 & Original & 678.93 & 5.43e-02 & 3.18 \\
256 & 4 & Tiled & 589.34 & 5.43e-02 & 3.66 \\
256 & 4 & Tiled+Aligned & 633.08 & 5.43e-02 & \textbf{3.41} \\
256 & 8 & Original & 398.76 & 5.43e-02 & 5.41 \\
256 & 8 & Tiled & 332.54 & 5.43e-02 & 6.48 \\
256 & 8 & Tiled+Aligned & 376.98 & 5.43e-02 & \textbf{5.72} \\
256 & 10 & Original & 412.83 & 5.43e-02 & 5.22 \\
256 & 10 & Tiled & 345.67 & 5.43e-02 & 6.24 \\
256 & 10 & Tiled+Aligned & 413.26 & 5.43e-02 & \textbf{5.22} \\
\hline
\end{tabular}
\end{table}

\subsection{三对角方程组求解完整数据}

表\ref{tab:tridiag_complete}展示了三对角求解器在所有测试规模下的完整性能数据:

\begin{table}[!htbp]
\centering
\caption{三对角方程组求解性能数据(加速比相对于Sequential)}
\label{tab:tridiag_complete}
\footnotesize
\begin{tabular}{llrrr}  % 合并"规模"和"大小"列
\toprule
\textbf{问题规模} & \textbf{并行方法} & \textbf{线程数} & \textbf{时间(ms)} & \textbf{加速比} \\
\midrule
\multicolumn{5}{c}{\textbf{8K(小规模灾难区)}} \\
\midrule
8K (8192)  & Sequential        & 1  & 0.23  & 1.00 \\
           & Brugnano          & 2  & 0.74  & \textbf{0.30} \\
           & Brugnano          & 4  & 0.97  & 0.23 \\
           & Brugnano          & 8  & 1.30  & \textbf{0.17} \\
           & Brugnano          & 16 & 1.80  & 0.12 \\
           & RecursiveDoubling & 2  & 0.77  & 0.34 \\
           & RecursiveDoubling & 4  & 1.34  & 0.19 \\
           & RecursiveDoubling & 8  & 1.83  & \textbf{0.14} \\
           & RecursiveDoubling & 16 & 2.52  & 0.10 \\
\midrule
\multicolumn{5}{c}{\textbf{128K(仍不适合并行)}} \\
\midrule
128K (131072) & Sequential        & 1  & 3.86  & 1.00 \\
              & Brugnano          & 2  & 6.83  & 0.61 \\
              & Brugnano          & 4  & 9.34  & 0.44 \\
              & Brugnano          & 8  & 7.62  & \textbf{0.54} \\
              & RecursiveDoubling & 4  & 12.43 & \textbf{1.15} \\
              & RecursiveDoubling & 8  & 13.76 & 1.04 \\
\midrule
\multicolumn{5}{c}{\textbf{1M(并行门槛)}} \\
\midrule
1M (1048576) & Sequential        & 1  & 40.17 & 1.00 \\
             & Brugnano          & 8  & 35.83 & \textbf{1.74} \\
             & Brugnano          & 10 & 33.21 & \textbf{1.88} \\
             & RecursiveDoubling & 10 & 47.76 & \textbf{2.04} \\
\midrule
\multicolumn{5}{c}{\textbf{4M(最佳并行规模)}} \\
\midrule
4M (4194304) & Sequential        & 1  & 163.28 & 1.00 \\
             & Brugnano          & 8  & 143.21 & \textbf{1.78} \\
             & RecursiveDoubling & 8  & 206.45 & 1.89 \\
\bottomrule
\end{tabular}
\end{table}

\textbf{数据分析:}
\begin{itemize}
\item 8K规模:所有并行方法都导致\textbf{5-8倍性能下降}
\item 1M规模:Brugnano首次获得正加速比(1.88×),RecursiveDoubling达到2.04×
\item 4M规模:最佳加速比仍不到2×,远低于其他算法
\item 线程数敏感:超过10线程性能立即下降
\end{itemize}

\section{关键代码实现}
\label{appendix:code}

本附录提供各算法的关键源代码实现。完整代码可在项目仓库获取。

\subsection{神经网络算子实现}

\subsubsection{卷积空间并行实现}

卷积算子的\texttt{collapse(2)}空间并行实现(\texttt{conv\_openmp\_new.cpp}):

\lstset{
    language=C++,
    basicstyle=\tiny\ttfamily,
    keywordstyle=\color{blue},
    commentstyle=\color{green!50!black},
    stringstyle=\color{red},
    numbers=left,
    numberstyle=\tiny,
    stepnumber=1,
    numbersep=5pt,
    showstringspaces=false,
    breaklines=true,
    frame=single,
    captionpos=b
}

\lstinputlisting[
    caption={卷积空间并行实现(collapse(2))},
    label={code:conv_space_parallel},
    linerange={1-50}
]{../operators/conv_openmp_new.cpp}

\textbf{关键优化点:}
\begin{itemize}
\item 使用\texttt{\#pragma omp parallel for collapse(2)}并行化输出特征图的H和W两个维度
\item 内层卷积循环保持串行,减少同步开销
\item 充分利用输出空间的独立性
\end{itemize}

\subsubsection{平均池化访存优化实现}

平均池化的访存优化版本(\texttt{avgpool\_openmp\_memory.cpp}):

\lstinputlisting[
    caption={平均池化访存优化实现},
    label={code:avgpool_memory_opt},
    linerange={1-60}
]{../operators/avgpool_openmp_memory.cpp}

\textbf{优化技术:}
\begin{itemize}
\item 行缓存复用:一次加载整行数据到缓存
\item 减少66\%的内存访问次数
\item 提高缓存命中率从50\%到70\%+
\end{itemize}

\subsection{Gauss-Seidel迭代法实现}

\subsubsection{2D红黑排序实现}

2D Gauss-Seidel的Tiled+Aligned优化版本(\texttt{gauss\_seidel\_2d\_tiled\_aligned.cpp}):

\lstinputlisting[
    caption={2D Gauss-Seidel Tiled+Aligned实现},
    label={code:gs2d_tiled_aligned},
    linerange={1-80}
]{../gauss_seidel/gauss_seidel_2d_tiled_aligned.cpp}

\textbf{三层优化:}
\begin{enumerate}
\item \textbf{红黑排序:}打破数据依赖,实现并行
\item \textbf{Tiling分块:}L1缓存块(32×32)+ L3缓存块(128×128)
\item \textbf{内存对齐:}64字节对齐,SIMD友好
\end{enumerate}

\subsubsection{3D红黑排序实现}

3D Gauss-Seidel的核心迭代循环(\texttt{gauss\_seidel\_3d\_tiled\_aligned.cpp}):

\lstinputlisting[
    caption={3D Gauss-Seidel红黑迭代核心},
    label={code:gs3d_core},
    linerange={50-120}
]{../gauss_seidel/gauss_seidel_3d_tiled_aligned.cpp}

\textbf{3D特殊处理:}
\begin{itemize}
\item 七点模板离散化
\item 三维红黑棋盘划分:$(i+j+k)\%2$
\item 三层Tiling:L1(16³) + L2(32³) + L3(64³)
\end{itemize}

\subsection{三对角方程组求解实现}

\subsubsection{Sequential Thomas算法}

标准Thomas算法串行实现(\texttt{sequential\_solver\_memtest.cpp}):

\lstinputlisting[
    caption={Sequential Thomas算法实现},
    label={code:thomas_sequential},
    linerange={1-50}
]{../new_tri/sequential_solver_memtest.cpp}

\textbf{算法特点:}
\begin{itemize}
\item 前向消元:$O(n)$严格串行依赖
\item 后向回代:$O(n)$严格串行依赖
\item 缓存友好:顺序访问,命中率$>$95\%
\end{itemize}

\subsubsection{Brugnano并行算法}

Brugnano区域分解并行实现(\texttt{openmp\_brugnano\_memtest.cpp}):

\lstinputlisting[
    caption={Brugnano并行算法实现},
    label={code:brugnano_parallel},
    linerange={1-100}
]{../new_tri/openmp_brugnano_memtest.cpp}

\textbf{四步算法:}
\begin{enumerate}
\item \textbf{区域划分:}分为$p$个独立子问题
\item \textbf{局部求解:}并行求解各子问题(忽略边界)
\item \textbf{规约系统:}构造$2p\times2p$边界修正系统(串行)
\item \textbf{边界更新:}并行更新所有边界值
\end{enumerate}

\subsubsection{Recursive Doubling算法}

递归倍增并行实现(\texttt{openmp\_recursive\_doubling\_memtest.cpp}):

\lstinputlisting[
    caption={Recursive Doubling算法实现},
    label={code:recursive_doubling},
    linerange={1-80}
]{../new_tri/openmp_recursive_doubling_memtest.cpp}

\textbf{算法特性:}
\begin{itemize}
\item $O(\log n)$步递归合并
\item 跳跃访问模式:stride=$2^k$
\item 缓存命中率低(<50\%),导致单线程比Thomas慢2.4×
\item 并行度高,但访存成为瓶颈
\end{itemize}

\subsection{测试脚本}

\subsubsection{批量性能测试脚本}

用于生成完整性能数据的PowerShell脚本(\texttt{test\_all.ps1}):

\begin{lstlisting}[language=bash, caption={批量性能测试脚本示例}, label={code:test_script}]
# 三对角求解器批量测试脚本
$sizes = @(8192, 16384, 131072, 1048576, 4194304)
$threads = @(1, 2, 4, 8, 10, 16, 20)
$programs = @("Sequential", "Brugnano", "RecursiveDoubling")

foreach ($size in $sizes) {
    foreach ($program in $programs) {
        foreach ($thread in $threads) {
            # 编译
            g++ -O3 -fopenmp -o solver.exe ${program}_solver.cpp
            
            # 运行10次取平均
            $times = @()
            for ($i=0; $i -lt 10; $i++) {
                $output = .\solver.exe $size $thread
                $time = [double]($output -match "Time: (\d+\.\d+)" | %{$Matches[1]})
                $times += $time
            }
            
            # 计算统计量
            $avg = ($times | Measure-Object -Average).Average
            $speedup = $baseTime / $avg
            
            # 输出到CSV
            "$size,$program,$thread,$avg,$speedup" | Add-Content results.csv
        }
    }
}
\end{lstlisting}

\section{实验环境详细配置}
\label{appendix:environment}

\subsection{硬件配置}

\begin{table}[!htbp]
\centering
\caption{测试平台硬件配置}
\label{tab:hardware_config}
\begin{tabular}{ll}
\hline
\textbf{组件} & \textbf{规格} \\
\hline
处理器 & AMD Ryzen 9 5950X \\
核心数 & 16 cores / 32 threads \\
基础频率 & 3.4 GHz \\
Boost频率 & 最高4.9 GHz \\
L1缓存 & 32KB I + 32KB D × 16 \\
L2缓存 & 512KB × 16 \\
L3缓存 & 64MB (2 × 32MB CCX) \\
内存 & 64GB DDR4-3200 (双通道) \\
内存带宽 & 51.2 GB/s (理论峰值) \\
操作系统 & Windows 11 Pro 22H2 \\
\hline
\end{tabular}
\end{table}

\subsection{软件环境}

\begin{table}[!htbp]
\centering
\caption{编译器和运行时配置}
\label{tab:software_config}
\begin{tabular}{ll}
\hline
\textbf{组件} & \textbf{版本/配置} \\
\hline
编译器 & GCC 11.2.0 (MinGW-W64) \\
OpenMP & 4.5 \\
编译选项 & \texttt{-O3 -fopenmp -march=native} \\
链接选项 & \texttt{-static} \\
数学库 & 无(手工实现) \\
调试工具 & gdb 11.2 \\
性能分析 & 手工计时(\texttt{chrono}) \\
\hline
\end{tabular}
\end{table}

\subsection{测试方法}

\textbf{计时方式:}
\begin{itemize}
\item 使用C++ \texttt{<chrono>}库的高精度计时器
\item 每个配置运行10次,取平均值
\item 舍弃首次运行(预热缓存)
\item 测量核心计算时间,不包括初始化
\end{itemize}

\textbf{负载控制:}
\begin{itemize}
\item 测试前关闭后台应用
\item 禁用CPU动态频率调整(固定最大频率)
\item 每次测试间隔5秒(冷却)
\item 多次测试标准差<5\%
\end{itemize}

\textbf{数据验证:}
\begin{itemize}
\item 所有并行结果与串行结果对比
\item 相对误差<$10^{-6}$视为正确
\item Gauss-Seidel收敛准则:残差<$10^{-4}$
\item 三对角求解精度:$\|Ax-b\|_2 / \|b\|_2 < 10^{-10}$
\end{itemize}

%% ========================================
%% 写在最后
%% ========================================
\section{写在最后}
\subsection{发布地址}
\begin{itemize}
    \item Github: \url{https://github.com/Jiazhen-Lei/SJTU_Course_Template_Latex}
    \item Overleaf:  \url{https://www.overleaf.com/latex/}
\end{itemize}

%%----------- 参考文献 -------------------%%
%在reference.bib文件中填写参考文献,此处自动生成

\reference


\end{document}